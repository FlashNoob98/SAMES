
\section{Segnali non stazionari}

Alcuni metodi utilizzati per misurare segnali non stazionari sono i PME, phasor
measurement unit, utilizzati e diffusi nei sistemi di trasmissione, in
alternativa se il segnale è monocomponente, ossia ha una frequenza fondamentale
e se ne voggliono misurare le fluttuazioni si può elaborare il segnale mediante
la trasformata di Hilbert in modulo e fase, la fase sarà una funzione nel tempo
con evoluzione lineare (a rampa), è talvolta interessante misurare il tasso di
variazione della frequenza effettuando la derivata seconda della fase, ROCOF
(rate of change of frequency).


\subsection{Rappresentazioni tempo-frequenza}
Si utilizzano le TFR ovvero le Time-Frequency representation tra cui la più
famosa la ``\textit{short-time Fourier transform}''

Si può dividere un brano in porzioni disgiunte e contigue, effettuando
l'analisi spettrale su una porzione del brano, si trova che il segnale
sinusoidale, con una frequenzza eventualmente instabile si trova una riga
spettrale centrata su quella riga di frequenzza, analizzando le porzioni
successive si vede se la riga spettrale si muove su frequenze puiù alte o più
basse.

Dato un segnale $s(t)$ la SFT fa corrispondere una funzione della frequeza e
del tempo
$$
s(t) \longrightarrow S(f,t)
$$

Ad un segnale nel tempo discreto si fa corrispondere la seguente sommatoria
$$
s(n) \longrightarrow S(k,nL) = \sum_{m=nL-M}^{nL+M} s(m)e^{\frac{-j2\pi k
m}{2M+1}\left(m-nL+M\right)}\ k=0,\dots , 2M
$$
equivale ad eseguire una DFT su $2M+1$ punti. È la lunghezza della porzione che
mi determina su quanti punti effettuare la somma.

Potrei usare una lunghezza arbitraria per ogni brano in funzione del parametro 
$L$.

Si snellisce il carico computazionale effettuando analisi in frequenza solo ogni
porzione temporale, ciò porta però ad avere una minore risoluzione temporale mentre
la risoluzione in frequenza dipende da $M$, maggiore è il numero di "bin" e migliore 
sarà l'analisi in frequenza. 

Al variare del parametro M si può modificare la risoluzione spettrale con un costo
computazionale diverso.

Con una diversa porzione di segnale al variare di $n$ si può considerare la seguente rappresentazione del brano
$$
S(k,nL) = \sum_{m=-M}^{M} s(m-nL) e^{-j\frac{2\pi}{2M+1}k(m+M)}
$$

Per una funzione che dipende da due variabili, la rappresentazione grafica è tipicamente
effettuata in uno spazio tridimensionale, su un asse sono rappresentate le $n$ misurazioni
spaziate di $L$, mentre sull'altro asse sono presennti le frequenze, si ha una rappresentazione
"a fette" dell'analisi in frequenza, tanti piani adiacenti con la proiezione delle cime.

%Libro Advanced DSP proakis paragrafi 2.4 e 5.6
\chapter{Sistemi di conversione dei segnali}
Il sistema di conversione è un sistema che ha in ingresso il segnale analogico e ne restituisce la versione 
digitalizzata, esistono svaraite architetture e sistemi che realizzano queesta conversoione, 
convertitori flash, monolitici o miltiplexati.

Il processo di conversione si può dividere in un processo di campionamento ed uno di quantizzazione,
talvola le architetture rispecchiano fisicamente quest approccio logico, spesso il circuito più usato è
il sample and hold, la quantizzazione invece è utilizzata dlla logice di coneg. Da un qunto di vista astratto,
la quantizzazione associa una variabile reale ad una naturale
$$
Q: \mathcal{R} \longrightarrow N : Q[x] =x_k \ \text{se } x\in \]_k 
= (d_k,d_{k+1)
$$
L'estensione dell'intervallo $\Delta = d_{k+1}-d_k$ è detto passo (o intervallo) di quantizzazione.
Se $x$ è espressa in unità di $\Delta$, allora le soglie saranno

$$
\begin{cases}
    d_k = k-\frac{1}{2}\\
    d_{k+1} = k+\frac{1}{2}
\end{cases}
$$
Si può supporre la quantizzazione $Q(x) \leq \text{rounding off}$

All'atto pratico i convertitori lavorano su un range di frequenze limitato, la curva di trasferimento 
è una gradinata crescente, le pedate anche sono identiche.
L'errore che si commette a causa della quantizzazione risulta limitata, al massimo metà del quanto.

Nella realizzazione dei convertitori bipolari, ci si può imbattere in due situazioni,il convertitore
mid-treat e a midrise, nel primo caso lo zero è proprio un punto di nullo, il convertitore restituisce un segnale nullo,
pari al rumore, la caratteristica è dissimetrica

Se il convertitore dovesse lavorare in una situazione di soovraccarico, allora il convertitore commetterebbe
un errore di quantizzazione superiore, diventando errore di sovraccarico, non limitato, ovviamente non si può superare
la tensione di isolamento del convertitore.

Se la caratteristica è dissimetrica, può essere resa simmetrica se si rinuncia ad utilizzare un codice,
utilizzando un numero dispari di codici pari e dispari, ad esempio $[-127,127]$ garantendo la 
presenza del codice 0.

Si possono effettuare varie codifiche dei quanti, ad esempio il binary offset oppure il bit di isegno se si lavora 
fino ad un massimo di , il primo bit più significativo implica... 

La codifica in complemento (ad uno), permette di sostituire gli zeri con gli 1, si numerano i quanti, con lo 
zero al bit più significativo, scrivo un numero positivo, viceversa sarà negativo.

La codifica più utilizzata è quella in complemento a due, si ottiene complementando la stringa e addizionando 
1.

Questa rappresentazione è la più utilizzata perchè l'ogetto più semplice all'interno della macchina
è l'addizionatore.

Quando si converte un segnale in forma digitale questo verrà degradato dall'errore di quantizzazione,
\dots \dots \dots

$$
\sigma^2_e \stackrel{\Delta}{=} \infint (e-\mu_e)^2 pdf(e) de 
= \int_{\frac{\Delta}{2}}^{\frac{\Delta}{2}} e^2 \frac{1}{\Delta} de = \frac{\Delta^2}{12} 
$$
