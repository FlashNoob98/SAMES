
Si ricorda la rappresentazione dei segnali periodici
$$
\kinfsum  c_ke^{j2k\pi f_0 t} \stackrel{\mathcal{F}}{\longrightarrow} \kinfsum
c_k \delta(f-kf_0)
$$
per ogni $k$ corrisponde uno spettro con modulo e fase pari al modulo e fase
del coefficiente $c_k$.

Si vuole ricavare la trasformata di Fourier del segnale coseno, si utilizza la
rappresentazione di Eulero:
$$
A\cos (2\pi f_0 t + \phi) = \frac{A}{2} \left( e^{j2\pi f_0 t + \phi} +
e^-{j2\pi f_0 t + \phi}\right)
$$
Trasformando si ottiene
$$
\frac{A}{2} \left( e^{j\phi} \delta(f-f_0) + e^{-j\phi}\delta(f+f_0) \right)
$$

Allo stesso modo per il seno
$$
A\sin(2\pi f_0 t) \stackrel{\mathcal{F}}{\longrightarrow}
\frac{A}{2j}\left( \delta(f-f_0) - \delta(f+f_0)\right)
$$


Per un segnale periodico la potenza è definita
$$
P = \lim_{T\to \infty} \frac{1}{T} \int_{T/2}^{T/2} |x(t)|^2 dt = \sum_k |c_k|^2
$$
L'ultima è nota in letteratura come identità di Parseval.

Allo stesso modo nel contesto dell'energia
$$
\epsilon_x = \infint |x|^2 dt = \infint |x(f)|^2 df
$$
lo stesso valore di energia può essere determinato nel dominio del segnale o
dominio del tempo.






\subsection{Convoluzione}
L'operatore \textit{convoluzione} è un operatore di tipo prodotto che coinvolge
due funzioni o due segnali o un segnale e una funzione.

Siano $u(t)$ e $v(t)$ il prodotto di convoluzione è così definito
\begin{equation}
 (u\ast v) (t) \stackrel{\Delta}{=} \infint u(\tau)v(t-\tau)d\tau =
\int_{+\infty}^{-\infty} v(t')u(t-t')(-dt') = \infint v(t')u(t-t') dt'
\end{equation}
con $t' = t-\tau$, $\tau = t-t'$ dunque $d\tau = -dt'$
