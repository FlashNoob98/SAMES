
\section{}
%Riassunto
Si è vista la finestra di Hamming, descritta dalla (inserisci ref), trasformata
secondo Fourier ottenendo un risultato pari a
$$
\frac{T}{2}e^{-j\pi fT}\sinc(fT) - \frac{T}{4}e^{-j\pi (f-\frac{1}{T})T}
\sinc((f-\frac{1}{T})T) - \frac{T}{4}e^{-j\pi(f+\frac{1}{T})T}
\sinc((f+\frac{1}{T})T))
$$

$$
= \frac{T}{4}e^{-j\pi fT}
(2\sinc(fT)-e^{j\pi}\sinc((f-\frac{1}{T})T) -
e^{-j\pi}\sinc((f+\frac{1}{T})T))
$$
svolgendo il modulo
$$
\frac{T}{4}(2\sinc(fT)+\sinc((f-\frac{1}{T})T) +
\sinc((f+\frac{1}{T})T))
$$

\textbf{Esercizio proposto}: Considerare una finestra di Hamming definita
simmetricamente rispetto all'origine $t \in [-\frac{T}{2},\frac{T}{2}]$
$$
h(t) = \frac{1}{2} + \frac{1}{2}\cos(\frac{2\pi t}{T})
$$

\textbf{Esercizio proposto}:
$$
FT[h(n)] = \sum_{n=0}^{N-1} h(n)e^{-j2\pi \nu n} = \frac{1}{2} D_n (\nu)
-\frac{1}{4} D_n (\nu - \frac{1}{N} - \frac{1}{4} D_N(\nu+\frac{1}{N})
$$
con
$$
D_N(\nu) = e^{-j\pi \nu (N-1)} \frac{\sin(\pi \nu N)}{\sin(\pi \nu)}
$$

\section{filtraggio di un segnale}
È un'operazione svolta per migliorare l'elaborazione del segnale a posteriori
dall'acquisizione, sia in tecniche analogiche che digitali.

Un passo spesso presente nella elaborazione del segnale è il filtraggio,
finalizzato alla riduzione del rumore, potrebbe essere inserito dall'atto
stesso della conversione analogica-digitale, anche introdotto dal convertitore
stesso.

Il filtraggio pu`o essere rappresentato da un sistema con la sua risposta
impulsiva $h(n)$ tale che produca un segnale
$$y(n) =
\sum_{m=-\infty}^{+\infty}x(n-m)h(m) = \sum_{m=-\infty}^{+\infty}h(n-m)x(m)
$$

All'atto pratico, il segnale di ingresso e di uscita, e la funzione di
trasferimento del sistema sono funzioni causali. Il prodotto di
convoluzione è commutativo.

Un semplice filtro può essere quello di media aritmetica così definita
$$
h(n) \frac{1}{N} \text{rect}_N(n)
$$


Il filtro ``scorre'' verso destra all'aumentare di $m$, ``scorrendo'' tutto il
segnale e mediando ogni volta i $N$ campioni precedenti al campione $m$.




$$
FT[\frac{1}{N}\text{rect}_N(n)](\nu) = \sum_{n=0}^{N-1} \frac{1}{N}e^{-j2\pi
\nu n} = \frac{1}{N} \sum_{n=0}^{N-1} x^n = \frac{1}{N} \frac{1-e^{-j2\pi
\nu N}}{1-e^{-j2\pi\nu}} \frac{\sin(\pi\nu N)}{\sin(\pi \nu)} = \frac{1}{N}
D_n(\nu)
$$


Moving average, è una categoria di filtri nella quale ricade anche la media
aritmetica, in alternativa una media triangolare è ancora una media mobile.

Altri filtri possibili sono chiamati Auto regressivi (AR), l'uscita è
$$
y(n) = a_0x(n) - b_1y(n-1) - \dots -b_{M-1} y(n-(M-1))
$$
Le uscite hanno memoria delle uscite agli istanti precedenti.

ARMA
$$
y(n) + b_1 y(n-1) + \dots  = a_0x(n) + a_1x(n-1) + \dots
$$

Un filtro con parte autoregressiva potrebbe avere memoria infinita e divergere.

\section{Z-trasformata}
Così definita
$$
X(z) = \sum_{n=0}^{+\infty} x(n)z^{-n}
$$
può essere vista come un'estensione al piano complesso della trasformata di
Fourier con $z = e^{-j2\pi \nu}$ o viceversa la trasformata di Fourier può
essere vista come una restrizione della trasformata di Fourier al cerchio di
raggio unitario.
Analogamente con la trasformata di Laplace, se l'ascissa di trasformabilità è
minore di zero, allora la restrizione della trasformata di Laplace all'asse
delle ordinate coincide con la trasformata di Fourier.


