
\section{Phasor Measurement Unit}
Le funzionalità di questi dispositivi sono un sottinsieme delle funzionalità
degli analizzatori di rete, utilizati per misurare la power quality.
Gran parte delle macchine elettriche sono alimentate da tensione sinusoidale,
caratterizzata da alcuni parametri che ne determinano la qualità, la tensione
ad esempio ha un valore nominale di 230$V_{rms}$, una frequenza nominale di $50
Hz$, nel 95\% del tempo il valore efficace è garantito dal gestore della rete.
Si può garantire inoltre una caratteristica sulla continuità del servizio, dal
punto di vista tecnico gli effetti riscontrabili sulla tensione di
alimentazione sono riconducibili a vari fenomeni:
\begin{itemize}
 \item Sag di tensione, ovvero un abbassamento del valore della tensione in
certi istanti di tempo
\item Swell, è il duale del Sag, è un incremento dei valori efficaci della
tensione, gli effetti di sag e swell possono alternarsi secondo uno schema
ripetitivo e dare origine al
\item flicker, ovvero uno sfarfallio, veniva percepito inizialmente con le
lampade, corrisponde ad una modulazione in ampiezza del segnale nel tempo.
Questo fenomeno è spesso dovuto alla presenza di carichi particolari come forni
ad induzione.
\item Interruzione, oltre ad un abbassamento si può avere una riduzione tale
che causa o viene considerata come un'interruzione della tensione, può essere
di breve durata, media o lunga durata.
La microinterruzione ha una durata inferiore alla durata del ciclo, se
coinvolge più cicli si parla di interruzione.
Alcuni macchinari sono più robusti di altri, ovvero hanno una certa inerzia nel
loro funzionamento, possono funzionare anche in presenza di interruzioni
dell'alimentazione, altri macchinari invece vanno in stallo.
Per l'utenza domiciliare la power quality non presenza un disservizio, per
l'utenza industriale invece un'interruzione del servizio può dar luogo anche a
gravi perdite economiche ed interruzioni del ciclo produttivo.
\end{itemize}

Gli strumenti di power quality possono anche essere installati nel punto da
misurare per periodi relativamente lunghi, al fine di misurare le performance
della rete nel tempo.

I PMU sono installati nel sistema di trasmissione di energia elettrica, sono
una rete distrumenti, sincronizzati tra loro e installati per capire se in una
certa area geografica per misurare se la tensione trasmessa dovesse presentare
degli sfasamenti, che potrebbero compromettere l'intera stabilità del sistema.

Per la sincronizzazione dei PMU si deve utilizzare un riferimento esterno, si
usa solitamente il segnale orario diffuso dai satelliti GPS.
Gli sfasamenti vanno contenuti per garantire la stabilità della rete.

Le versioni più evolute dei PMU permettono di effettuare anche l'analisi
armonica.

Si vogliono estrarre i valori di misura acquisiti dagli strumenti, esiste un
front-end analogico, dopodichè il senale si digitalizza poi viene elaborato.
La tensione di rete è di tipo sinusoidale ma utilizzando tale modello si
trascurerebbero tanti parametri.

Il rumore può essere filtrato da un sistema digitale o
analogico.
I due segnali ottenuti, chiamati x e y, combinati tra loro danno origine al
``phase unwrap''.

%%%%% BUCO GALATTICO


Il metodo dei battimenti e l'elaborazione descritta dallo schema ha il
vantaggio di isolare il termine delle variazioni della fase rispetto ad un
riferimento atteso, con i battimenti si effettua la proiezione di un fasore su
un piano con due riferimenti ortogonali con coseno e seno, se la frequenza di
riferimento è proprio quella del segnale, si ottiene la proiezione di un
fasore, altrimenti c'è la fluttuazione, viceversa il termine $A$ si corregge
automaticamente eseguendo il rapporto nell'algoritmo, le variazioni di ampiezza
si compensano.

Se si prendono i segnali $X$ e $Y$ e si estrae la radice $X^2$ e $Y^2$ si
ottiene un termine costante $A/2$ e un termine pari ad $A/2a(n)$.

La frequenza istantanea è stata definita come
$$
f(t) \stackrel{\Delta}{=} \frac{1}{2\pi} \frac{d}{dt}\Psi(t)
$$
con $\Psi(t) = 2\pi f_0t - \phi(t)$


Il PMU può essere utilizzato per la misura di frequenza istantanea, non sono
però progettati per lavorare su ampi range.

Un segnale può essere rappresentato in una forma quasi ridondante, con un
corrispondente segnale complesso
$$
x(t) \rightarrow \tilde{x}(t)) = x(t) + jx_H(t)
$$
con la trasformata di Hilbert così definita
$$
x_H(t) \stackrel{\Delta}{=} \frac{1}{\pi}\infint  \frac{x(\tau)}{t-\tau}d\tau
$$
La trasformata di Hilbert restituisce la componente in quadratura rispetto al
segnale
$$
x(t) = A\cos(2\pi f_0 t) \longrightarrow \tilde{x}(t) = Ae^{j2\pi f_0 t} =
A\left[ \cos(2\pi f_0 t) + j\sin(2\pi f_0 t) \right]
$$

Valutando la fase istantanea e derivando il segnale si può facilmente ottenere
la frequenza istantanea da questa rappresentazione del segnale.
$$
f(t) = \frac{1}{2\pi} \frac{d}{dt} \left[
\text{tan}^{-1}\left(\frac{x_H(t)}{x(t)}\right) \right]
$$

