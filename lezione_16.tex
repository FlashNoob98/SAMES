
Algoritmi di regressione linearizzati

Si potrebbero avere dati che possono essere rappresentati da una legge
esponenziale
$$
Y_k = Ae^{ax_k} + \varepsilon_k
$$
Sembra un problema non lineare per ricavare $A$ ed $a$ ma se si passa ai
logaritmi
$$
\ln y_k = \ln(Ae^{ax_k}) = \ln A + ax_k
$$

Se il modello è invece del tipo
$$y_k = A\sqrt{x_k + B} + \varepsilon_k
$$
eseguendo il quadrato
$$
y_k^2 = A^2(x_k + B) = A^2x_k + A^2B
$$

Spesso i problemi di regressione lineare non sono sostanziali, possono
essere ricondotti a problemi lineari con semplici passaggi.

Alcuni casi più ostici:
$$
y_k = \frac{ax_k + b}{x_k + c} + \varepsilon_k
$$
In questo caso non si trovano metodi per linearizzare direttamente la funzione.

Assumendo
$$
x_k \neq -c
$$
allora $$
x_ky_k + cy_k = ax_k + b
$$
$$
ax_k + b - cy_k = x_ky_k
$$

Andrebbe aggiunto il termine di errore $\varepsilon_k(x_c + c) $, aumenterebbe
all'aumentare dell'ingresso, di conseguenza questo approccio non funziona.


\section{Sinusoidal fitting a quattro parametri}
Si vogliono adattare i dati acquisiti ad un modello sinusoidale
$$
y = A_c\cos(2\pi\nu n) + B_0\sin(2\pi\nu 0) + C_0 + \varepsilon_n
$$

$$
\bar{\bar{D_0}}\bar{x} = \bar{y} \rightarrow \bar{x} = (\bar{\bar{D_0}}^T
\bar{\bar{D}}^{-1}\bar{\bar{D_0}}^T \bar{y}
$$

Si fissa un punto di riferimento
$$
\nu = \nu_0 \Rightarrow \hat{x_0}^T = (A_0,B_0,C_0)
$$

$$
y = A_1\cos(2\pi\nu_0 n) + B_1\sin(2\pi\nu_0 n) + C_1 + 2\pi
n(-A_0\sin(2\pi\nu_0 n) + B_0\cos(2\pi\nu_0 n))\Delta\nu_1
$$

$$
\bar{\bar{D}}_1 \bar{x}_1 = \bar{y} \Rightarrow \bar{x}_1 =
(\bar{\bar{D}}_1^T\bar{\bar{D}}_1)^{-1} \bar{\bar{D}}_1^T \bar{y}
$$

Si può iterare il procedimento e considerare i parametri con pedice 1 come i
nuovi parametri di partenza per il calcolo di parametri più raffinati.

La matrice $D_1$ ha la seguente forma
$$
\begin{bmatrix}
\cos(2\pi\nu_0 1) & \sin(2\pi\nu_0 1) & 1 & (-2\pi1 A_0
\sin(2\pi\nu_01)+2\pi1\cos(2\pi\nu_01))\\
\vdots & \vdots & \vdots & \vdots \\
\cos(2\pi\nu_0 M) & \sin(2\pi\nu_0 M ) & 1 & (-2\pi M A_0
\sin(2\pi\nu_0 M)+2\pi M\cos(2\pi\nu_0 M))
\end{bmatrix}
$$
La procedura è convergente, dunque le correzioni decrescono con le iterazioni.

Questo metodo è una variante di Newton Rhapson, si dimostra essere convergente.



Il fitting è molto buono al centro del brano nel punto in cui inizia
l'algoritmo, se c'è un errore sulla frequenza questo si manifesta
progressivamente allo scostarsi dal punto centrale, il vettore
$\bar{\varepsilon} = \bar{y} - \bar{\bar{D_0}}\bar{x}$ si annulla nel punto
centrale e tende ad aumentare gradualmente, quando questo vettore resta
confinato in una fascia uniforme, allora si è eseguito un buon fitting.

\section{Algoritmo di Goërtzel}
È un algoritmo di tipo autoregressivo, serve a stimare l'ampiezza di un segnale
sinusoidale di frequenza approssimativamente nota, è come eseguire la DFT in un
solo punto, anzichè eseguirla su tutto il segnale, elabora infatti il segnale
in \textit{streaming}, analizzando punto per punto man mano che vengono
acquisiti.
$$
y(n) = x(n) + e^{j2\pi \nu_0} y(n-1)
$$
Veniva utilizzato nel sistema telefonico per la trasmissione del numero,, in
sistemi per segnalamento breve, nei sistemi denominati DTMF (Dual Tone Multi
Frequency), in questi sistemi si definisce un alfabeto di simboli e a ciascun
simbolo si associa una coppia di frequenze.
\begin{table}[h]\centering
\begin{tabular}{c | c | c| c| c}
 $f_0$ &1209 &1336 & 1477 &1633 \\ \hline
 697 & 1 & 2 & 3 & A \\ \hline
 770 & 4 & 5 & 6 & B \\ \hline
 852 & 7 & 8 & 9 & C \\ \hline
 941 & $\textasteriskcentered$ & 0 & \# & D \\ \hline
\end{tabular}
\end{table}
Per ogni numero è associata una coppia di frequenze che generano una sequenza
di Burst con la loro sovrapposizione.

L'algoritmo produce il valore della DFT della sequenza in corrispondenza di un
certo BIN, il parametro $\nu_0$ è $\frac{l}{N}$, dopo una sequenza di $N+1$
passi dunque
$$
y(N) = DFT_N\{x_(n)\}(l)
$$

$$
y(n)  = x(n) + e^{j2\pi \nu_0} (x(n-1) + e^{j2\pi\nu_0}y(n-2))
$$
$$
y(n) = x(n) + e^{j2\pi\nu_0}(x(n-1)+e^{j2\pi\nu_0}
(x(n-2)+e^{j2\pi\nu_0}y(n-3)))
$$
$$\begin{aligned}
y(n) &= \sum_{m=0}^{\infty}x(n-m)e^{j2\pi\nu_0 m} = \{l = n-m\} =
\sum_{l=n}^{\cancel{-\infty}0}x(l)e^{j2\pi\nu_0(n-l)} = \\
&=(\sum_{l=0}^{n} x(l)e^{-j2\pi\nu_0 l})e^{j2\pi\nu_0 n} \| y(N) =
e^{j2\pi\frac{l}{N}N}\left( \sum_{l=0}^{N-1} x(l) e^{-j2\pi\nu_0 l} + X(N)\cdot
1 \right)
\end{aligned}$$

Si dovrebbero eseguire dei prodotti tra numeri complessi, si preferisce
interporre una variabile di stato eseguendo la trasformata $Z$
$$\begin{aligned}
Y(Z) &= X(Z) + e^{j2\pi\nu_0} Z^{-1} Y(Z) \\
&(1-e^{j2\pi\nu_0} Z^{-1})Y(z) = X(z)
\end{aligned}$$
$$
\frac{Y(Z)}{X(Z)} = \frac{1}{1-e^{j2\pi\nu_0 Z^{-1}}}\cdot
\frac{1-e^{-j2\pi\nu_0}Z^{-1}}{1-e^{-j2\pi\nu_0}Z^{-1}} =
\frac{1 - e^{-j2\pi\nu_0} Z^{-1}}{1-2\cos(2\pi\nu_0)Z^{-1} + Z^{-2}}
$$
La funzione di trasferimento del sistema può essere vista come
$$
\frac{Y(Z)}{X(Z)} = \frac{S(Z)}{X(Z)}\frac{Y(Z)}{S(Z)}  =
\frac{1}{1-2\cos(2\pi\nu_0)Z^{-1} + Z^{-2}} (1-e^{-j2\pi\nu_0}Z^{-1})
$$
Si introduce la variabile di stato $s(n)$
$$
s(n) = x(n) + 2\cos(2\pi\nu_0)s(n-1) - s(n-2)
$$

$$
y(N) = s(N) - e^{-j2\pi\nu_0}s(N-1)
$$

Le condizioni iniziali per la variabile sono
$$
\begin{cases}
s(-1) = 0 \\
s(-2) = 0
\end{cases}
$$
Così facendo non bisogna implementare l'aritmetica dei numeri complessi ma si
lavora solo con numeri reali.

Infine ci si interessa al modulo quadro di $y(n)$
$$
|y(N)|^2 = y(N)y^*(N) = \left(s(N)-e^{-j2\pi\nu_0}s(n-1)\right)(s(N) -
e^{j2\pi\nu_0}s(N-1)) = s(N) + s(N-1) - 2\cos(2\pi\nu_0)s(N)s(N-1)
$$



