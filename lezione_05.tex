
La rappresentazione spettrale è equipollente alla rappresentazione alternativa
che sarebbe possibile fornire con la rappresentazione analitica.

La rappresentazione più comune è quella mediante l'utilizzo degli spettri, in
ampiezza e fase, l'ampiezza si assume convenzionalmente positiva, un'eventuale
inversione di segno viene riportata nello spettro delle fasi.

Per un segnale periodico sussiste la seguente rappresentazione a coefficienti
complessi
$$
x(t) = \kinfsum c_k e^{j 2\pi k f_0 t}
$$
Si può dunque passare da una rappresentazione monolatera ad una bilatera, a
frequenze ``negative'' corrispondono valori delle fasi di $c_k$
antisimmetrici, di segno opposto dato che $c_k = c_k^*$.

Viceversa lo spettro di ampiezza sarà simmetrico rispetto all'asse verticale,
le ampiezze però saranno dimezzate ($|c_k| = A_k/2$).

Sono sensate dunque anche le frequenze negative che compaiono nella
rappresentazione bilatera.

La sequenza dei coefficienti complessi è Hermitiana, ossia sono sequenze
complesse che rispettano una simmetria pari nel modulo e dispari nella fase, la
loro somma dà sempre una coppia di termini reali.


Segnale replica dell'impulso di durata $\tau$ e ampiezza unitaria.


$$
x(t)=AR_Rep_{T_0}\{P_\tau(t)\}
$$
Se la durata degli impulsi diminuisce, deve aumentare l'altezza, si ricorda che
l'integrale dell'impulso è sempre unitario.

Si esegue l'integrale dopo aver calcolato il limite.
$$
+{\tau\to0} \text{ rep}_{T_0} \left\{\frac{1}{\tau_z}\right\} = \lim_{\tau \to
0}
\left[ \frac{1}{\tau} \kinfsum  P_\tau(t-kT_0) \right] =
\hat{x}(t)
$$
dunque
$$
\int_{\hat{k}T_0-\varepsilon}^{\hat{k}T_0-\varepsilon} \hat{x}(t)dt = 1
$$
Il treno di impulsi opportunamente scelto converge ad una funzione
generalizzata denominata \textit{Delta di Dirac}.

La delta gode della seguente proprietà di \textit{campionamento}
ossia
$$
\infint x(t)
\delta(t-t_0) dt = x(t_0)
$$

Si analizza lo spettro del treno campionatore, si dice che assuma uno spettro
``a pettine''.

$$
\frac{1}{\tau} \text{ rep}_{T_0} \{P_\tau(t)\} = \sum{k-\infty}^{+\infty} c_k
e^{j2\pi k f_0} = ... = \sum f_0e^{j2k\pi f_0}
$$

Sono tutti multipli alla stessa ampiezza.

Lo spettro a pettine può essere chiamato con la sommatoria
$$
\kinfsum \delta(t-kT_0)
$$


\section{Analisi spettrale di segnali non periodici - Trasformata di Fourier}
Lo spettro di un segnale descritto da una certa funzione $x(t)$ è ciò che si
ottiene valutando la trasformata di Fourier.

Si consideri un segnale periodico, può essere rappresentato in una forma
compatta ed elegante in una forma trigonometrica con coefficienti complessi
$c_k$.

Se il segnale non è periodico? Si assuma un segnale non periodico con un
periodo $T_0$ estremamente grande, allora $f_0\to 0 $ dunque il contenuto
spettrale si addensa in una zona più piccola.

Si riscrive il coefficiente al variare della frequenza. Si potrebbe
rappresentare il segnale mediante i contributi pesati in frequenza.

Si ottiene la definizione o equazione di \textit{sintesi} di Trasformata di
Fourier
$$
F[x(t)] = \infint X(f) e^{j2\pi ft} df
$$

Viceversa l'equazione di analisi della trasformata di Fourier
$$
F^{-1}[X(f)] = \infint x(t) e^{-j2\pi ft}dt
$$

Mediante le formule di Eulero è possibile scomporre la trasformata,
$$
X(f) = \infint x(t) e^{-j 2\pi f t} dt =
\infint x(t) \cos(2\pi f t) dt - j \infint
x(t) \sin (2\pi f t) dt
$$

La rappresentazione nel dominio della frequenza è una delle varie
rappresentazioni spettrali, una qualsiasi trasformazione biunivoca permette di
ottenere una rappresentazione spettrale di un segnale.

Trasformata del segnale $x(t)$ nella variabile $f$
$$
F[x(t)](f)
$$
una qualunque funzione $a(x)$ può essere trasformata nella rappresentazione
$A(y)$ dunque $F[a(x)](y)$.
$$
F[a(x)](y) = \infint a(x)e^{-j2\pi y x} dx
$$

L'antitrasformata è in realtà molto simile alla trasformata, cambia il segno
dell'esponenziale
$$
x(t) = \infint X(f) e^{j2\pi f t } dt
$$
sono cioè quasi la stessa cosa, ossia la trasformata di Fourier si definisce
quasi \textbf{iperpotente}.
Ad esempio l'operatore ``opposto'' (o negato) è un operatore iperpotente, se
applicato due volte ritorna alla condizione di partenza.
La trasformata di Fourier non è completamente iperpotente, va cambiato il segno
all'esponenziale, in caso contrario se si applica due volte la trasformata di
Fourier ad un segnale $x(t)$ si ottiene una versione ``ribaltata''
$$
F \left[F [x(t)]\right] = x(-t)
$$
Applicata a segnali pari è invece iperpotente, il ribaltamento di un
segnale simmetrico non ha effetto sul risultato.

\subsection{Esempio di trasformata di un segnale}
Si consideri un segnale transitorio denominato finestra rettangolare, indicato
con $A\text{rect}_T(t)$, si considera solitamente centrata in 0, di ampiezza
$T$, calcoliamone la trasformata di Fourier per esercizio nella variabile $f$.
$$\begin{aligned}
F[A\text{rect}_T(t)](f) &= \infint A\text{rect}_T(t)e^{-j2\pi f t} dt =
A\int_{-T/2}^{T/2} e^{-j2\pi f t } dt = \left.\frac{A}{-j2\pi f}e^{-j2\pi ft
}\right|_{-T/2}^{T/2} = \\
&= \frac{A}{-j2\pi f } \left( e^{-j\cancel{2}\pi f T/\cancel{2}} -
e^{j\cancel{2}\pi f T/\cancel{2}}
\right) = \frac{A}{\pi f}\sin(\pi fT) = AT\text{sinc}(fT)
\end{aligned}$$

Si potrebbe rappresentare lo spettro del modulo della funzione, potrebbe essere
una funzione complessa ma poiché la funzione da trasformare è pari allora la
sua trasformata è reale e conserva la parità.
%rappresenta modulo sinc, si ottiene l'andamento a ``lobi''
C'è un lobo principale in zero dove la funzione sinc è prolungabile per
continuità e l'ampiezza della finestra è pari alla sua altezza ($AT$), gli zeri
si hanno in corrispondenza dei valori interi dell'argomento, ovvero quando
$f=n/T$ con $n\in \mathbb{N}$.

Dunque
$$
A\text{rect}_T(t) \stackrel{\mathcal{F}}{\longrightarrow} AT\text{sinc}(fT)
$$
$$
AB\text{sinc}(tB) \stackrel{\mathcal{F}}{\longrightarrow} A\text{rect}_B(f)
$$


Lo spettro di una sinc è uno spettro uniforme di un intervallo di semiampiezza
$B$, ossia sono contenute tutte le frequenze in maniera uniforme di ampiezza
$2A$ nella rappresentazione fisica monolatera da $0 \text{ a } B$, questo
segnale non è fisicamente realizzabile, ha uno spettro a banda limitata detto
``brick-wall''
%Inserisci grafico da sinc a brick

Si immagini di avere un segnale in continua di ampiezza unitaria,
trasformandolo secondo Fourier si ottiene il seguente integrale
$$
1 \stackrel{\mathcal{F}}{\longrightarrow}\infint e^{-j2\pi f t} dt = \quad ?
$$
Come potrei calcolare quest'integrale? si procede diversamente calcolando la
trasformata dell'impulso
$$
\delta(t) \stackrel{\mathcal{F}}{\longrightarrow} \infint\delta(t) e^{-j2\pi
ft}dt = 1
$$
dunque la trasformata della funzione costante è proprio $\delta(t)$.


Con un generico segnale periodico rappresentato nella sua forma generale con i
coefficienti complessi si ottiene, eseguendo la trasformata
$$\begin{aligned}
\kinfsum c_ke ^{j2\pi k f_0 t}
\stackrel{\mathcal{F}}{\longrightarrow} &\infint
\kinfsum c_ke ^{-j2\pi(f -  k f_0) t} dt =
\kinfsum c_k \infint e^{-j 2\pi (f-kf_0)t} dt = \\
&= \kinfsum c_k \delta(f-kf_0)
\end{aligned}$$
I segnali periodici hanno dunque spettro ``a righe'', il seno campionatore ha
uno spettro a pettine, le ampiezze delle righe sono costanti.

