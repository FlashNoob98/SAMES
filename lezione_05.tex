
La rappresentazione spettrale è equipollente alla rappresentazione alternativa
che sarebbe possibile fornire con la rappresentazione analitica.

La rappresentazione più comune è quella mediante l'utilizzo degli spettri, in
ampiezza e fase, l'ampiezza si assume convenzionalmente positiva, un'eventuale
inversione di segno viene riportata nello spettro delle fasi.

Per un segnale periodico sussiste la seguente rappresentazione a coefficienti
complessi
$$
x(t) = \sum_{k=-\infty}^{+\infty} c_k e^{j 2\pi k f_0 t}
$$
Si può dunque passare da una rappresentazione monolatera ad una bilatera, a
frequenze ``negative'' corrispondono valori delle fasi di ``$c_k$
antisimmetrici, di segno opposto dato che $c_k = c_k^*$.

Viceversa lo spettro di ampiezza sarà simmetrico rispetto all'asse verticale,
le ampiezze però saranno dimezzate ($|c_k| = A_k/2$).

Sono sensate dunque anche le frequenze negative che compaiono nella
rappresentazione bilatera.

La sequenza dei coefficienti complessi è Hermitiana, ossia sono sequenze
complesse che rispettano una simmetria pari nel modulo e dispari nella fase, la
loro somma dà sempre una coppia di termini reali.


Segnale replica dell'impulso di durata $\tau$ e ampiezza unitaria.


$$
x(t)=AR_Rep_{T_0}\{P_\tau(t)\}
$$
Se la durata degli impulsi diminuisce, deve aumentare l'altezza, si ricorda che
l'integrale dell'impulso è sempre unitario.

Si esegue l'integrale dopo aver calcolato il limite.
$$
+{\tau\to0} \text{ rep}_{T_0} \left\{\frac{1}{\tau_z}\right\} = \lim_{\tau \to
0}
\left[ \frac{1}{\tau} \sum_{k=-\infty}^{+\infty}  P_\tau(t-kT_0) \right] =
\hat{x}(t)
$$
dunque
$$
\int_{\hat{k}T_0-\varepsilon}^{\hat{k}T_0-\varepsilon} \hat{x}(t)dt = 1
$$
Il treno di impulsi opportunamente scelto converge ad una funzione
generalizzata denominata \textit{Delta di Dirac}.

La delta gode della seguente proprietà di \textit{campionamento}
ossia
$$
\int_{-\infty}^{+\infty} x(t)
\delta(t-t_0) dt = x(t_0)
$$

Si analizza lo spettro del treno campionatore, si dice che assuma uno spettro
''a pettine``.

$$
\frac{1}{\tau} \text{ rep}_{T_0} \{P_\tau(t)\} = \sum{k-\infty}^{+\infty} c_k
e^{j2\pi k f_0} = ... = \sum f_0e^{j2k\pi f_0}
$$

Sono tutti multipli alla stessa ampiezza.

Lo spettro a pettine può essere chiamato con la sommatoria
$$
\sum_{k=-\infty}^{+\infty} \delta(t-kT_0)
$$


\section{Analisi spettrale di segnali non periodici}
Lo spettro di un segnale descritto da una certa funzione $x(t)$ è ciò che si
ottiene valutando la trasformata di Fourier.

Si consideri un segnale periodico, può essere rappresentato in una forma
compatta ed elegante in una forma trigonometrica con coefficienti complessi
$c_k$.

Se il segnale non è periodico? Si assuma un segnale non periodico con un
periodo $T_0$ estremamente grande, allora $f_0\to 0 $ dunque il contenuto
spettrale si addensa in una zona più piccola.

Si riscrive il coefficiente al variare della frequenza. Si potrebbe
rappresentare il segnale mediante i contributi pesati in frequenza.

Si ottiene la definizione o equazione di \textit{sintesi} di Trasformata di
Fourier
$$
F[x(t)] = \int_{-\infty}^{+\infty} X(f) e^{j2\pi ft} df
$$

Viceversa l'equazione di analisi della trasformata di Fourier
$$
F^{-1}[X(f)] = \int_{-\infty}^{+\infty} x(t) e^{-j2\pi ft}dt
$$
