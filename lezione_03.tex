\section{Interfaccia VXI IEEE 1155}
Un'interfaccia che raccolse una certa attenzione fu la tecnologia \textbf{VXI},
si perse l'aspetto stand-alone dello strumento dato che i sistemi VXI sono molto
compatti e ospitati in un cestello, sono strumenti su scheda elettronica,
possono essere alloggiati in qualche slot, non c'è alcuno spazio negli slot per
un'interfaccia completa, si perde la caratteristica stand-alone.

Il cestello tramite il back plain alimenta le schede, l'interazione può
avvenire solo mediante l'interfaccia VXI, tramite righe di comando per
richiedere determinate funzioni.
Il vantaggio è avere una stazione di misura particolarmente compatta, la
strumentazione su scheda riesce a scambiare flussi dati tra i vari moduli ad
una velocità maggiore rispetto all'interfaccia IEEE 488.
I bus sono molto corti, a volte sono piste stampate sul back plain, c'è la
capacità di trasferire i dati in maniera efficiente.

Standard similari al VXI sono stati \textbf{PXI}, cestelli ancora più compatti,
concepiti alla stessa maniera.
Si è persa l'interfaccia degli strumenti e le unità non sono più stand-alone,
durante l'esecuzione del processo di misura non c'è alcun feedback su cosa stia
accadendo, si vedono al massimo dei LED lampeggiare, l'utilizzatore deve avere
un livello professionale più alto.
Il tecnico ha dunque un costo più alto.
In più la stessa strumentazione ha dei costi elevati, quindi non è semplice
sostituire gli strumenti più vecchi, si tende dunque a restare legati alla
soluzione, anche più vecchia, sulla quale sono stati investiti più soldi.

Gran parte delle PMI adotta infatti ancora lo standard GPIB con strumentazione
stand-alone.
Il successo è dunque stato limitato rispetto allo standard GPIB.

\section{Interfaccia $I^2C$ e $SPI$}
L'acronimo \textbf{$I^2C$} sta per Inter Integrated Circuit, sono protocolli di
basso livello, consentono la comunicazione tra moduli elettronici integrati,
dei chip, comparsi circa negli anni '80, comparse prima SPI sviluppata da
Motorola, la sezione che ora se ne occupa è FreeScale Semiconductor.
Il microprocessore Motorola 6800 fu molto diffuso.

Il protocollo SPI è un protocollo master-slave, saranno presenti sulla scheda
elettronica un modulo master e uno o più moduli slave.
Sono presenti più linee, le più comuni sono
\begin{itemize}
\item SLCK Linea di clock
\item MOSI Master Output Slave Input, linea uilizzata dal master
\item MISO Master Input Slave Output linea utilizzata dagli slave per
comunicare con il master
\item SS Selezione dello Slave, il master indica con quale unità vuole
colloquiare
\end{itemize}
La comunicazione è full-duplex, l'uno può parlare all'altro e viceversa
contemporaneamente, viceversa si avrebbe una comunicazione half-duplex se si
usasse una sola linea.

Per ogni comunicazione sono necessarie 4 linee che (eccetto la SS) possono
essere condivise tra i vari slave, deve invece essere necessaria una linea SS
sul chip master per ogni slave che si vuole collegare.

La linea di SLCK e SS è ad intero appannaggio del master, viceversa le altre
due linee possono essere controllate dagli slave.
Non può esistere una comunicazione diretta tra due slave.

Il protocollo non vincola la frequenza del clock, il master potrebbe decidere di
lavorare con un clock più alto op più basso, a seconda del chip disponibile.
(?) Il master si adatta allo slave facendolo lavorare alla sua frequenza
indicata.

La tipologia di comunicazioni si dividono in
\begin{itemize}
 \item Clock Polarity (0-1)
\item Clock Phase (0-1)
\end{itemize}

Il chip slave può essere impostato per configurare il dato sul fronte di
salita, e leggerlo sul fronte di discesa.
L'attività sulle linee MOSI e MISO sono sincrone.


Il Master può parlare di volta in volta con uno slave, non c'è la
possibilità di una comunicazione multipla


\section{Protocollo I2C}
Il protocollo I2C è stato utilizzato molto ad esempio per il dispositivo
Arduino.

Esistono solo due linee nel protocollo I2C, una linea di clock ed una per il
trasferimento dati, il pilotaggio delle linee avviene in maniera simile alla
linea GPIB ma si utilizzano dei mosfet al posto dei BJT.
Il protocollo è multi-master e consente di collegare tra loro non solo coppie di
dispositivi ma un insieme di \textit{n} dispositivi collegati sulla stessa
coppia.
Ogni dispositivo può diventare master sul BUS che è quindi una risorsa
condivisa.

Chi vuol parlare sul BUS dà un segnale di START che corrisponde ad un evento
anomalo, dato che lo standard prevede che la linea dati possa essere
configurata solo quando il clock è basso, la transizione della linea dati
quando il clock è alto è un evento anomalo; in generale il protocollo prevede
che la linea dati possa essere configurata solo a clock basso. Allo stesso modo
il segnale di STOP è un evento anomalo con una configurazione della linea dati
SDA con il clock SCL alto.

Ogni dispositivo presenta un indirizzo a 7bit, esiste anche un meccanismo a 10
bit, la pluralità di dispositivi è elevatissima, fino a 128 in teoria con i 7
bit, abilitando la comunicazione a 10 bit si raggiungono invece numeri enormi.

Il modulo che dà il segnale di START identifica il dispositivo slave con il
quale vuole parlare mediante il suo indirizzo, l'ottavo bit indica se il master
vuole ascoltare o leggere dallo slave, 1 voglio leggere, 0 voglio scrivere.
Dopo aver trasferito i dati (8bits) si ha il bit di risposta dell'ascoltatore
che invia il bit ACK di acknowledge, attesta la linea, la
logica non è negata quindi linea attestata è pari al livello logico 0 e
viceversa con linea rilasciata.

Si trasmettono i dati per 8 bit e chi ascolta invia un bit di ACK ogni 8 bit
ricevuti, chi parla dopo aver trasmesso il suo messaggio rilascia il BUS con
l'evento di STOP, rilasciando la linea dati quando il clock è alto.
Il bit ACK è sempre fornito da chi ascolta a prescindere se questo sia il
master o lo slave.

Nel caso in cui il master sia in ascolto questo può terminare la comunicazione
inviando un messaggio di ACK rilasciando la linea e non attestandola (bit 1)
successivamente si invia l'evento di STOP sul BUS.
Anche se il messaggio non è stato trasferito completamente, il master potrebbe
tornare in ascolto del dispositivo successivamente.

Potrebbe capitare che due unità decidano di parlare simultaneamente sulla linea
che è comunque in logica wired-or, se un dispositivo attesta la linea questa va
giù per tutti, entrambe le periferiche avviano la configurazione della linea
dati per inviare l'indirizzo della periferica slave con la quale vogliono
parlare. Se gli indirizzi sono diversi prevale allora uno dei due.

Ad esempio un master vuole parlare con lo strumento ad indirizzo 1

\verb|0000001|

mentre un altro vuole parlare con il dispositivo ad indirizzo 3

\verb|0000011|

Quando si trasmette in contemporanea il sesto bit il dispositivo che voleva
comunicare con quello ad indirizzo 3 si rende conto che non può rilasciare la
linea che invece è stata attestata dal dispositivo che comunicava con indirizzo
1, dunque si mette in IDLE e capisce che non deve continuare la comunicazione.
Si dice dunque che il dispositivo ad indirizzo 1 ha la priorità su quello ad
indirizzo 3.

Se ancora entrambi i master volessero parlare con lo stesso dispositivo il
controllo della linea avverrebbe sulla trasmissione dei dati, uno dei due
dispositivi master si rende conto in ogni caso di dover porsi in attesa.

Nella configurazione a 10bit si inviano prima quattro 1 per inviare un comando
riservato e far capire ai dispositivi che si sta utilizzando l'indirizzamento a
10 bit, 8 in un byte e i restanti due nel byte precedente.


Il CLOCK è variabile, si può lavorare in modalità slow mood o fast mood o
ancora fast mood Plus, o ancora una modalità a 3.4Mb/s, non tutte le
periferiche supportano queste velocità, il master attiva la comunicazione al
valore più alto possibile (10kHz 400kHz o 1 MHz).
Esiste un meccanismo di CLOCK stretching che consente ad una periferica di
rallentare il clock. Anche la periferica richiamata dunque può intervenire
sulla linea di clock e gestirla, dunque se una periferica che lavora
nominalmente a 10kHz dovesse aver problemi e rallentare potrebbe comunque
gestire la linea di clock e raggiungere ancora il sincronismo.

Ancora oggi si utilizzano entrambe le soluzioni SPI e I2C.


