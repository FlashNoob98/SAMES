\chapter{Teorema del campionamento uniforme}

Se $s(t)$ è un segnale a banda limitata, detto $S(f)$ lo spettro del segnale,
se il segnale è a banda limitata $B$ allora
$$
S(f) = 0 \ \forall|f| > B
$$

Se la frequenza di campionamento è maggiore di $2B$ allora è possibile
ricostruire fedelmente la $s(t)$ andando ad interpolare i campioni
e la formula di ricostruzione è la seguente
$$
s(t) = \kinfsum s(k) \text{sinc}(f_st-k)
$$


Si raccoglie un numero discreto di campioni nell'intervallo di tempo a distanza
$T_s$

\section{Dimostrazione}
Si riprende il discorso tra il rapporto tra campionamento e replicazione, nel
dominio del tempo sia un segnale
$$
s(t) \cdot \kinfsum \delta(t - kT_s)
$$
nel dominio della frequenza si ha la replicazione del segnale originario
$$
S(f) \ast f_s \kinfsum \delta(f-kf_s))
$$

Nelle repliche del segnale nel dominio della frequenza si avranno degli
``alias'' a distanze multiple di $f_s$, il cui supporto sarà ampio proprio
$2B$, di conseguenza avendo $f_s>2B$ si garantisce la non intersezione tra le
diverse repliche del segnale, altrimenti si avrebbe il cosiddetto fenomeno
dell'aliasing o dell'interferenza tra le repliche, di conseguenza il segnale
non potrà essere ricostruito a partire dai suoi campioni.



Sia dato un segnale campionato $s(k)$ allora si potrà ottenere il segnale
$s(t)$ andando ad antitrasformare lo spettro del segnale campionato
$$
s(t) = F^{-1} [ f_s \kinfsum S(f-kf_s)\cdot \frac{1}{f_s} \text{rect}_{f_s}(f)
](t)
$$

dove $\text{rect}_{f_s}$ è una finestra rettangolare di guadagno uniforme e di
ampiezza $f_s$, per compensare il fattore di guadagno prodotto dall'operazione
di campionamento.

Applicando la regola del prodotto di convoluzione all'interno della trasformata
$$
s(t) = F^{-1} [ f_s \kinfsum S(f-kf_s) ](t) \ast F^{-1} [ \frac{1}{f_s}
\text{rect}_{f_s}(f)
](t)
$$

Il primo termine è il campionamento del segnale, ossia
$$
s(t)\cdot \kinfsum\delta(t-kT_s) \Rightarrow \sum_{n=-\infty}^{+\infty} s(n)
\delta(t-nT_s)
$$
il secondo è la sinc$(f_sT)$, svolgendo la convoluzione si ricava
$$\begin{aligned}
s(t) &= \infint \sum_{n=-\infty}^{+\infty} s(n) \delta(\tau - nT_s) \cdot
\text{sinc}(f_s(t-\tau))d\tau = \\
&= \sum_{-\infty}^{+\infty} \delta(\tau - nT_s) \text{sinc}(f_s(t-\tau))d\tau
=\\
&= \sum_{n=-\infty}^{+\infty}
\end{aligned}
$$


$$\begin{aligned}
&A\text{rect}_T(t) \stackrel{F}{\longrightarrow} AT\text{sinc}(fT)\\
&AB\text{sinc}(Bt)
\stackrel{\stackrel{F}{\longrightarrow}}{\stackrel{\longleftarrow}{F^{-1}}}
A\text{rect}_B(f)
\end{aligned}$$


\section{Trasformazione di Hilbert}
Si effettua una trasformazione dallo spettro reale allo spettro Hermitiano
rispettando le seguenti regole di simmetria
$$\begin{aligned}
|S(f)| = |S(-f)| \\
\angle S(f) = -\angle S(-f)
\end{aligned}$$

Lo spettro $S(f)$ contiene infatti informazioni ridondanti, possono essere
eliminate se si elimina il contenuto spettrale sull'asse negativo, si ipotizza
ad esempio di moltiplicare lo spettro reale per una funzione gradino al fine di
ottenere la rappresentazione monolatera
$$
\tilde{S}(f) = 2S(f)1(f)
$$
il termine 2 compensa l'energia rimossa dal segnale nella rappresentazione
bilatera.

La funzione di Heaviside è così definita
$$
1(f) = \begin{cases}
1 \text{ per } f> 0 \\
\frac{1}{2} \text{ per } f=0\\
0 \text{ per } f<0
\end{cases}
$$
oppure con il seguente limite
$$
1(f) = \lim_{\varepsilon\to0} \left[ \frac{1}{2} + G_\varepsilon(f) \right]
$$

La rappresentazione analitica nel tempo della funzione di Heaviside
$$
\tilde{s}(t) = F^{-1} [ 2S(f)1(f) ](t) = 2s(t) \ast F^{-1}[1(f)](t)
$$
si svolge l'antitrasformata
$$
F^{-1} 1(f)](t) = F^{-1}\left[ \frac{1}{2} \right](t) +  \lim_{\varepsilon \to
0}F^{-1}[G_\varepsilon (f)] (t)
$$
la derivata di $G_\varepsilon$ sarà:
$$
G'_\varepsilon (f) = \frac{1}{2\varepsilon} \text{rect}_\varepsilon(f)
$$
$$\begin{aligned}
G'_\varepsilon(f) &= \frac{d}{df} G_\varepsilon(f) = \frac{d}{df} \left(
g_\varepsilon(t) e^{-2\pi ft} dt \right) = \\
&\infint
g_\varepsilon(t) \frac{\partial}{\partial f} \left( e^{-j2\pi ft} \right)dt =
\infint \left[ -j2\pi ft g_\varepsilon(t) \right]e^{-j2\pi ft} dt
\end{aligned}$$

$$
G'_\varepsilon (f) \stackrel{F^{-1}}{\longrightarrow} -j2\pi t g_\varepsilon(t)
$$
$$
\frac{d}{df} \stackrel{F^{-1}}{\longrightarrow} -j2\pi t
$$
