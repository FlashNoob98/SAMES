
\section{Filtro di Hilbert}
Non attenua le altre frequenze



Vabbè rivediti la lezione registrata


\section{Segnali a tempo continuo}
Quando si opera con dei segnali e si vuole valutare gli effetti di un filtro,
si valuta l'uscita del filtro, si considera un segnale $s(t)$ e si misura
l'uscia al filtro, si può misurare il segnale però solo in un tempo limitato
$T$, dunque dato un segnale ad esempio coseno, non osservo il segnale coseno ma
il segnale coseno moltiplicato per la ``$\text{rect}_T$'', il fatto che il
segnale si osservi per un intervallo di durata finita $T$ ma il segnale è
definito per tutto l'asse dei tempi, allora nella analisi il segnale in esame
sarà solo una porzione del segnale originale, bisogna tener conto di un
``fattore finestra'' per rendicontare che si osserva il segnale per un tempo
finito.

Trasformando secondo Fourier si ottiene uno spettro pari alla trasformata del
primo fattore convoluto la trasformata del secondo
$$
&F[A\cos(2\pi f_0 t) \ast F[\text{rect}_T(t)](f) = \\
&=\frac{A}{2}[\delta(f-f_0) + \delta(f+f_0)] \ast T\text{sinc}(fT) = \\
&= \frac{AT}{2} \text{sinc}((f-f_0)T) + \frac{AT}{2} .....
$$



