\chapter{Metodi per il controllo statistico del processo}
% Fonti: www.biblioteca.dieti.unina.it
% Taylor & Francis
% Spinger
% Statistical quality control (process)
% Wiley
% Mc Graw Hill
% Douglass Montgomery
% Introduction to statistical Quality Control

In tutte le industrie e le aziende che producono prodotti o erogano servizi 
c'è sempre più attenzione verso il concetto di \textbf{qualità} di un servizio
o un prodotto, esistono le carte di qualità con l'elenco degli standard che le 
aziende si impegnano a garantire. In campo industriale c'è un gruppo di persone 
che si interessano di tenere pronti degli strumenti per realizzare controlli 
sul prodotto, tra cui le persone operative che eseguono prove e collaudi sul 
prodotto. È un'azione essenziale monitorare il processo produttivo affinchè 
esso evolva in modo efficiente, lo stesso monitoraggio è utile per determinare 
criticità o migliorie applicabili al processo.
Una migliore qualità del processo implica quasi sempre una migliore qualità del 
prodotto.
La qualità è un attributo associato ad un processo o un prodotto ma non è una 
grandezza fisica, dunque non si misura alla stessa maniera.

È possibile scindere la qualità in:
\begin{itemize}
    \item Intrinseca: È vista come la rispondenza del prodotto a specifiche 
    dimensionali, può essere determinata con misure dirette sul prodotto
    \item Estrinseca: Coinvolge la percezione dell'utente, non è facilmente 
    quantificabile
\end{itemize}
Con il tempo si è affermato il concetto di associare la qualità estrinseca a 
quella intrinseca, si ritiene che ci sia una correlazione implicita tra le due, 
ragion per cui la qualità estrinseca si correla a quella intrinseca.

Se si vuole effettuare una misura di qualità ha senso dunque soffermarsi su 
quella intrinseca, la qualità può allora essere vista come un insieme di 
grandezze di riferimento misurabili, una serie di indici di qualità sui vari 
aspetti del prodotto, osservabili e quantificabili, nei casi più semplici si 
considererà spesso una grandezza misurabile in modo oggettivo, verranno 
comunque forniti esempi di qualità associata a grandezze quantificabili con più 
difficoltà.

\newpage
I metodi non sono diversi da quelli itilizzati nell'indagine speculativa di 
tipo scientifico o tecnico, questi richiedono di:
\begin{enumerate}
    \item Collezione dei dati
    \item Validazione dei dati
    \item Presentazione dei dati, viene vista come una fase intermedia, 
    propedeutica all'elaborazione, si può ricorrere a diversi tipi di 
    rappresentazione che mettano o meno in evidenza particolari analogie o 
    caratteristiche del processo
    \item Elaborazione dei dati
    \item Presentazione dei risultati, viene stigmatizzato ciò che si è 
    individuato nella fase di elaborazione dei dati, eventualmente con una 
    controverifica
\end{enumerate}

Verranno esposti alcuni metodi di presentazione dei dati utilizzati.
Solitamente questi vengono rappresentati in tabelle o grafici ma esistono 
ulteriori metodi come istogrammi, aerogrammi ecc... sono molto semplici e 
facilmente intuibili.

\section{Stem \& Leaf Plot}
Un particolare strumento di presentazione può essere il \textit{diagramma stelo 
e foglie} o \textit{Stem \& Leaf Plot}
In molte aziende o enti vengono svolte azioni di monitoraggio del processo, 
soprattutto in molte aziende ospedaliere convenzionate con il servizio 
sanitario nazionale vengono riportati gli indici di prestazione della 
struttura, come ad esempio il tempo di permanenza in terapia intensiva, è un 
parametro che influisce molto sui costi.
L'università monitora costantemente i parametri che caratterizzano le 
prestazioni o l'efficienza delle attività di formazione e ricerca, anche i 
reparti amministrativi hanno delle misurazioni di performance dei processi, uno 
di questi potrebbe essere ad esempio il tempo di rimborso delle missioni al 
personale.
È importante effettuare queste misurazioni per migliorare ed efficientare i 
processi.

Si considera in questo esempio il tempo di rimborso, il parametro è il numero 
di giorni necessario al rimborso e il tempo effettivamente trascorso prima 
della notifica alla banca di avvenuto bonifico.
Si suddividono i dati in quadrimestri:
\begin{table}[h]
    \centering
    \begin{tabular}{c | c | c | c}
      & 1Q & 2Q & 3Q \\ \hline
    1 & 30 & 22 & 31 \\ \hline
    2 & 24 & 24 & 20 \\ \hline
    3 & 31 & 25 & 17 \\ \hline
    4 & 26 & 25 & 17 \\ \hline
    5 & 35 & 26 & 13 \\ \hline
    6 & 25 & 25 & 21 \\ \hline
    7 & 24 & 24 & 24 \\ \hline
    8 & 30 & 25 & 23 \\ \hline
    9 & 11 & 23 & 22 \\ \hline
    10 & 7  & 28 & 20 
    \end{tabular}
\end{table}

Nel diagramma stelo e foglie si avrà:
\begin{table}[h]
    \centering
    \begin{tabular}{c|l}
        S & L \\ \hline
        0 & 7 \\ \hline
        1 & 1 3 \\ \hline
        1 & 9 7 \\ \hline
        2 & 4 4 2 4 4 4 3 0 1 4 3 2 0 \\ \hline
        2 & 6 5 5 6 5 5 8 \\ \hline
        3 & 0 1 0 \\ \hline
        3 & 5 \\ \hline
        4 & 4
    \end{tabular}
\end{table}

Viene poi riordinato.
\begin{table}[h]
    \centering
    \begin{tabular}{c| c|l}
      &  S & L \\ \hline
    1  &  0 & 7 \\ \hline
    3  &  1 & 1 3 \\ \hline
    5  &  1 & 7 9 \\ \hline
    (13) &  2 & 0 0 1 2 2 3 3 4 4 4 4 4 4  \\ \hline
    12  &  2 & 5 5 5 5 6 6 8\\ \hline
    5  &  3 & 0 0 1 \\ \hline
    2  &  3 & 5 \\ \hline
    1  &  4 & 4
    \end{tabular}
\end{table}
Il numero in parentesi indica il numero di foglie perchè si "scollina" rispetto 
ai dati.
Visivamente si vede la disersione dei dati e la tendenza centrale, è poi 
possibile ispezionare facilmente i dati per valutare i precentili, il k-mo 
percentile è il dato indicizzato da $k/100(N + 0.5 )$ con $N$ la numerosità 
complessiva delle osservazioni.
Si calcola ad esempio il 10mo percentile, è il valore indicizzato da 3.5, sarà 
il valore tra il terzo e il quarto rilievo ordinati in maniera crescente, 
dunque tra il 10 
$$
10^\circ \rightarrow 15\ g
$$
la mediana invece
$$
50^\circ \rightarrow 24\ g
$$
Il primo quartile:
$$
25^\circ \rightarrow 21\ g
$$
il terzo quartile:
$$
75^\circ \rightarrow 26\ g
$$

Un ulteriore indice è l'IQR (range interquartile), è la differenza tra il terzo 
e il primo quartile
$$
IQR = 26 - 21 = 5\ g
$$
è un indice della dispersione dei dati.
La distribuzione è abbastanza piccata, la maggior parte dei valori sono 
addensati attorno al dato mediano, l'IQR è anche piccolo dunque si può fare 
affidamento che in circa tre settimane una pratica viene conclusa.

\section{Box plot}
Il \textbf{Box Plot} è tipicamente un rettangolo rappresentabile in un 
diagramma degli assi cartesiani, ha gli estremi compresi tra il primo e terzo 
quartile, viene anche rappresentata la mediana da una riga centrale, si 
rappresenta un box per ogni quadrimestre, si vede subito come si sia stati 
particolarmente efficienti nel secondo quadrimestre.
È un caso? I box plot vengono tipicamente tracciati con i "baffi" (wiskers), 
senza questi si perderebbero i dati che non rientrano nel quartile, si avrà la 
rappresentazione, mediante i wiskers, il prolungamento verso i valori minimo e 
massimo.

Più recentemente si è diffuso l'uso dei box-plot modificati, hanno una 
lunghezza dei baffi standardizzata e pari ad 1.5 volte l'IQR, questo fa sì che 
se ci fossero degli elementi all'esterno di questo range viene rappresentato 
singolarmente mediante una crocetta o un asterisco e chiamati \textit{outlier}, 
ovvero dati ``anomali" che probabilmente non fanno statistica.

Implementano criteri per implementare o validare i dati o effettuare una 
cernita nella valutazione delle prestazioni. In questo caso la grandezza non è 
misurabile in quanto grandezza fisica, nonostante sia comunque un parametro 
oggettivo.


Talvolta è possibile riportare i valori delle singole osservazioni in un 
diagramma cartesiano, ssi vede ad esempio un certo andamento in un anno di 
esercizio, confrontato con il logbook dei dati degli anni precedenti si può 
determinare la caratterisica del processo di erogazione del servizio, si 
possono individuare ad esempio delle cause stagionali, nel periodo estivo i 
tempi si allungano.

\section{Istogramma}
È possibile vedere come si disperde l'indice di qualità mediante l'istogramma, 
i dati vengono ripartiti in ``classi'', il numero di classi viene solitamente 
scelto mediante la regola di Sturge $n_c = 1 + \log_2 (N)$ nel caso di numerosi 
dati oppure $n_c = \sqrt{N}$ per $N$ piccolo, nessuno vieta di utilizzare un 
numero diverso di calssi.
I dati vengono classificati mediante un intervallo $\Delta = \frac{max-min}{n_c}
$ in ognuno di questi intervalli si riporta il numero di occorrenze di valori 
che rientrano in questo intervallo, si rappresenta dunque un bar-plot con le 
occorrenze.
I numeri sono solitamente riportati in termini relativi, sommando tutte le 
occorrenze si ottiene proprio $N$.
Talvolta si rappresentano istogrammi con classi di estensione non uniforme, in 
tal caso anzichè riportare il numero di occorrenze si riporta il numero di 
occorrenze diviso il $\Delta_k$, in tal caso sarà la somma delle aree dei 
rettangoli a fornire il valore unitario o $N$ se l'istogramma è fornito in 
termini relativi o meno.

Talvolta l'istogramma viene rappresentato nella forma di istogramma 
``cumulativo'' ovvero nella rappresentazione cumulativa delle occorrenze, in 
maniera incrementale fino a raggiungere l'ordinata $N$ o 1.

\section{Indici}
Gli indici utilizzati nella rappresentazione dei dati sono visti come variabili 
aleatorie, possono essere di tipo discreto o continuo. Per variabili aleatorie 
discrete o continue sono forniti concetti di distribuzione di probabilità e 
funzione densità di probabilità.
\begin{table}[h]\centering
    \begin{tabular}{c | c|c}
             & Media & Varianza \\ 
    Discreta & $\mu = \sum_i x_i p(x_i)$ & $ \sigma^2_{x_i}= \sum_{i} (x_i -\mu)^2 p(x_i)$ \\ \hline
    Continua & $\mu = -\int_{-\infty}^{+\infty} pdf(x) dx$ &  $\sigma_x^2 = \int_{-\infty}^{+\infty} (x-\mu)^2 pdf(x)dx $ 
    \end{tabular}
\end{table}

Non ha senso definire nelle variabili continue una probabilità per ogni valore 
nell'intervallo, il valore della probabilità si ricava integrando gli estremi 
nell'intervallo di probabilità continua.
L'integrale della funzione di densità di probabilità fornisce la distribuzione 
cumulativa della variabile continua.


Se la distribuzione di probabilità è uniforme $p(x_i) \text{ uniforme } p(x_i) = \frac{1}{N}\ \forall\ i$ dunque la media $ \mu = \frac{1}{N} \sum_i x_i$
e la varianza
$\sigma_{x_i}^2 = \frac{1}{N} \sum_i (x_i - \mu)^2 $



\subsection{Esempio di analisi di un processo con due dadi}

Si riesce a deterimanre la distribuzione di probabilità dei risultati 
ottenibili con il lancio di due dadi?
Gli esiti, intesi come punteggi sono tutti i valori compresi tra 1 e 12.
$$
x_i \ \text{punteggi } = \{2, \ldots, 12\}
$$
Si calcola il numero delle possibili combinazioni realizzabili 
$$\begin{aligned}
p(2) &= \frac{1}{36} \quad \{1,1\} \\
p(3) &= \frac{2}{36} \quad \{1,2\} \cup \{2,1\} \\
p(4) & = \frac{3}{36} \quad \{1,3\} \cup \{3,1\}\cup\{2,2\}\\
p(5) & = \frac{4}{36} \quad \{1,4\} \cup \{4,1\} \cup \{2,3\} \cup \{3,2\} \\
p(6) &= \frac{5}{36} \quad \{1,5\}\cup\{5,1\}\cup\{2,4\}\cup\{4,2\}\cup\{3,3\}  \\
p(7) & = \frac{6}{36} \quad \{1,6\}\cup\{6,1\}\cup\{2,5\}\cup\{5,2\}\cup\{3,4\}\cup\{4,3\}\\
p(8) &= \frac{5}{36} \quad \{2,6\}\cup\{6,2\}\cup\{3,5\}\cup\{5,3\}\cup\{4,4\}  \\
p(9) & = \frac{4}{36}\quad \{3,6\}\cup\{6,3\}\cup\{4,5\}\cup\{5,4\} \\
\end{aligned}
$$

Volendo condensare la distribuzione di probabilità in un'espressione analitica
si potrebbe affermare che:
$$
p(x_i) = \frac{1}{6} - \frac{|x_i-7|}{36}\quad x_i=2,\ldots,12
$$

%Inserisci nella lezione successiva
Definizione di \textbf{fattoriale}
$$
n! \stackrel{\Delta}{=} n(n-1)\cdot(n-2)\cdot\ldots\cdot2\cdot1
$$
Per un insieme di $n$ elementi, il fattoriale rappresenta il numero di 
possibili permutazioni degli elementi.
È facile dimostrare per induzione il numero di possibili permutazioni al 
variare di $n$.

