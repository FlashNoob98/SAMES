
\section{Analisi spettrale di segnali a tempo discreto}
Mediante l'utilizzo della DFT, i risultati ottenuti a causa del fenomeno della
\textit{dispersione spettrale} possono indurre delle stime di parametri di
frequenza e ampiezza delle singole componenti viziate da errore di
polarizzazione per la stima di frequenza e scallop loss per la stima di
ampiezza.

Riguardo la polarizzazione esiste un approccio teorico mediante l'algoritmo di
Buleman che permette di compensare l'errore, la correzione è corretta se il
segnale contienne una sola riga spettrale. Nel caso reale nessun segnale è mai
composto da una sola frequenza, dunque tale algoritmo dà spesso risultati
insufficienti nel caso pratico.

La polarizzazione può essere ridotta se si utilizza un algoritmo empirico che
svolge una media pesata dei bin nell'intorno del picco relativo.

Per compensare l'errore di scallop loss invece si utilizzano varie soluzioni,
riferite tipicamente come \textbf{finestratura}, l'analisi si svolgerà
nuovamente mediante la variabile a tempo continuo e poi applicata a quelle a
tempo discreto, preso il segnale $x(t)$ questo viene moltiplicato per una
finestra $w(t)$, tale operazione era stata svolta per descrivere l'osservazione
di un segnale in un intervallo finito.

$$
x(t)\cdot w(t) = x(t)\cdot \text{rect}_T(t)
$$

Il fatto che nell'analisi si abbia una finestra rettangolare fa sì che la
dispersione spettrale assuma particolari connotati, nell'analisi condotta il
segnale era di tipo coseno, descritto utilizzando le formule di Eulero, lo
spettro del segnale consisteva in una riga a frequenza $+f_0$ ed una a
frequenza $-f_0$. L'operazione di finestratura, con la finestra rettangolare
generava uno spettro a lobi, il parametro $T$, ampiezza della finestratura, era
inversamente proporzionale dall'ampiezza dei lobi.

Va considerato lo spettro che si ottiene mediante il prodotto di convoluzione
dello spettro del segnale e della trasformata di Fourier nel dominio della
frequenza della funzione rettangolare, pari alla sinc$(fT)$.

La dispersione spettrale dipende dal tipo di finestra, si potrebbe introdurre
una finestra esplicita per interpretare i dati nel modo in cui lo spettro del
segnale viene pesato introducendo dei coefficienti di peso, o condizionati
tramite una finestra esplicita.

Scegliendo una finestra differente dalla funzione rettangolare si può ridurre
l'errore di scallop loss, ad esempio una finestra con una forma spettrale
caratterizzata da un lobo con una ``testa più piatta'' permette di ottenere una
pari amplificazione per un intervallo più ampio di frequenze nell'intorno di
$f_0$, in caso di campionamento non coerente no si commette un errore della
stessa entità che si commetterebbe senza utilizzare questo tipo di finestratura.

Si consideri la seguente finestra:
$$W(t) =\begin{cases}
 & \frac{1}{2} - \frac{1}{2}\cos(2\pi \frac{t}{T})\ \text{per } 0<t<T\\
&0 \ \text{per } t<0 \text{ oppure } t>T
\end{cases}
$$
%Inserisci grafico funzione

Questa finestra pesa maggiormente i valori al centro della finestra di
osservazione, dando meno rilievo ai valori alle estremità della finestra,
prende il nome dello studioso che la introdusse nell'analisi spettrale,
chiamata \textit{Finestra di Hanning}, rappresentabile con $h(t)$, in forma
compatta si può scrivere con
$$
h(t) = \left(\frac{1}{2}-\frac{1}{2}\cos\left( 2\pi \frac{t}{T}
\right)\right)\cdot \text{rect}_T\left(t-\frac{T}{2}\right)
$$

Vanno analizzate le caratteristiche nel dominio della frequenza, si svolge la
trasformata di Fourier:
$$\begin{aligned}
H(f) &= \infint h(t) e^{-j2\pi ft} dt = \int_0^T
\left(\frac{1}{2}-\frac{1}{2}\left( \frac{e^{j\frac{2\pi}{T}t}
+e^{-j\frac{2\pi}{T}t}}{2} \right) \right)e^{-j2\pi ft} dt =\\
& = \int_0^T \frac{1}{2}e^{-j2\pi ft} dt = \left.\frac{1}{(-j2\pi f)2}e^{-j2\pi
ft} \right|_0^T = \frac{1}{2\pi f} \frac{e^{-j2\pi fT} -1}{-2j} =\\
&= \frac{1}{2\pi f} e^{-j\pi fT} \left( \frac{e^{j\pi f T}-e^{-j\pi f T}}{2j}
\right) = \frac{1}{2\pi f}e^{-j\pi fT} \sin(\pi f T) = \frac{T}{2} e^{-j\pi f
T} \text{sinc}(fT) \\
&- \int_0^T \frac{1}{4} e^{j\frac{2\pi}{T}t} e^{-j2\pi ft}dt = -\frac{1}{4}
\int_0^T e^{-j2\pi (f-\frac{1}{T})t} dt
\end{aligned}
$$

raggruppando entrambi i termini si ricava
$$
H(f) = \frac{T}{2}e^{-j\pi fT}\text{sinc}(fT) - \frac{T}{4} e^{-j\pi
(f-\frac{1}{T})}\text{sinc}(f+\frac{1}{T})T) %RIVEDIIIIII
$$

Si confrontano le due caratteristiche spettrali di entrambe le finestre
%38:40


La finestra di Hanning ha le caratteristiche spettrali ricercate, dati i
termini aggiuntivi infatti, si è generato uno schema di interferenza per cui i
lobi secondari diventano meno importanti, la potenza del segnale viene comunque
dispersa ma è dispersa nell'intorno della frequenza centrale, si è peggiorata
la dispersione a banda stretta ma migliorando notevolmente la dispersione a
banda larga.

%52:00

%%Pausa


Se il segale $X(f)$ viene campionato, e se è campionato ``bene'', allora il
segnale avrà uno spettro che sarà una versione ``periodicizzata'' dello spettro
del segnale originale, effettuando una operazione di scalatura, si otterrebbe
lo spettro ottenibile a partire dalla frequenza di campioni prelevata  da
quella frequenza di campionamento.


%%%


%%

\section{Zero padding}
Il costo che si paga per eseguire lo zero padding è un costo puramente
computazionale, eseguire tale operazione significa che acquisito un segnale
discreto, ovvero una sequenza N-esima di valori, si può prolungare la sequenza
con degli zeri, ottenendo una sequenza più lunga, prolungando il segnale senza
alterarlo.

Investigando le caratteristiche spettrali del segnale si ottiene l'effetto di
avere una funzione periodica tra 0 ed 1.




