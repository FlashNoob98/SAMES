
\section{ARMA}
Elaborazione a media mobile autoregressivo, in un sistema autoregressivo,
l'uscita risente della sollecitazione all'istante corrente, dunque $x(n)=a_0
x(n)$

Un sistema a media mobile MA si descrive con
$$
y(n) + b_1y(n-1) + \dots + b_My(n-M) = a_0 x(n) + \dots + a_n x(n-M)
$$
Un sistema AR invece
$$
y(n) = a_0 x(nn) - b_1y(n-1) + \dots + b_My(n-M)
$$


È utile utilizzare la Z-trasformata
$$
X(z) = \sum_{n=0}^{\infty} x(n)z^{-n}
$$
per segnale causale tale che per $n<0 \rightarrow x(n)=0$.

La Z-transform trasforma l'elemento di ritardo in una moltiplicazione algebrica,
è facile dimostrare che
$$
Z[x(n-L)](z) = z^{-L}X(z)
$$
dim:
$$
Z[x(n-L)] = \sum_{n=0}^{+\infty} x(n-L)z^{-n} = \{l=n-L\} =
\sum_{l=-L}^{+\infty} x(l) z^{-(l+L)} \stackrel{\text{causalità}}{=}
z^{-L}\sum_{l=0}^{\infty}x(l)z^{-l} = Z^{-L}X(z)
$$

Applicando la trasformata al segnale ARMA
$$
Y(z)(1+b_1z^{-1} + \dots + b_Mz^{-M}) = X(z)(a_0+a_zZ^{-1} + \dots + a_nz{-N})
$$
Dunque si ricava facilmente la funzione di trasferimento di un sistema discreto
eseguendo la divisione tra i due termini
$$
\frac{Y(z)}{X(z)} = \frac{a_0 + a_1z^{-1} + \dots + a_Nz^{-N}}{1 + b_1z^{-1} +
\dots + b_Mz^{-M}}
$$
La funzione di trasferimento è rappresentata in una sua forma generale, è
possibile utilizzare altre forme come la fattorizzata scomponendo il polinomio
in prodotto di zeri e poli.
È una funzione olomorfa, razionale fratta e regolare in un intervallo del piano
complesso.

La Z-trasformata è biunivoca, si può ottenere la funzione di partenza eseguendo
l'integrale su un circuito chiuso
$$
x(n) = \frac{1}{2\pi j} \oint X(z)z^{n-1}dz
$$

Sia $\Omega \subseteq \mathbf{C},\ z_z \in \Omega$ il dominio $\Omega$
regolare in $C$ allora
$$
f(z) = \sum_{n=0}^{\infty} c_n(z-z_0)^n
$$
In campo complesso una funzione è olomorfa se è analitica e viceversa.

Si vuole conoscere il valore di una funzione in un punto $z_0$, esso sarà
l'integrale lungo un circuito chiuso $C(z_0)$ che circonda il punto $z_0$
$$
f(z_0) = \frac{1}{2\pi j} \oint_{C(z_0)} \frac{f(z)}{z-z_0} dz
$$


Se la funzione dovesse presentare dei punti di discontinuità (ad es in $z_0$)
allora vale il seguente risultato: la funzione avrà una parte olomorfa e una
parte formata ancora da una serie con esponenti negativi
$$
f(z) = \sum_{n=9}^{+\infty} c_n(z-z_0)^n + \sum_{n=1}^{+\infty}
c_{-n}(z-z_0)^{-n} = \sum_{n=-\infty}^{+\infty} c_n(z-z_0)^n
$$
Se il punto $z_0$ coincide con l'origine allora la funzione sarà
la serie di Laurent
$$
f(z) = \sum_{n=-\infty}^{+\infty} c_nz^n
$$

Valutando la \ref{integrale} in un punto di discontinuità, non si otterrà il
valore della funzione, non definita in quel punto ma un valore comunque finito,
per semplicità si svolge l'esempuo di una funzione olomorfa in $C$ tranne che
nell'origine
$$
\frac{1}{2\pi j} \oint_{C_1(0)} \frac{f(z)}{z} dz = c_0
$$
Si ottiene proprio il coefficiente della serie di Laurent, sostituendo la serie
nell'integrale
$$
\frac{1}{2\pi j} \int \frac{\sum_{n=-\infty}^{+\infty} c_nz^n}{z}dz \dots
$$
È comodo parametrizzare il circuito $C_1(0): z=e^{j2\pi \nu}$ con $\nu\in [0,1[$
mentre il differenziale diventa $dz = d(e^{j2\pi \nu} = j2\pi e^{j2\pi \nu}
d\nu $
$$\begin{aligned}
\dots &= \frac{1}{\cancel{2\pi j}} \sum_{n=-\infty}^{+\infty} c_n \int_0^1
\frac{e^2\pi j\nu n}{\cancel{e^2\pi j \nu}} \cdot \cancel{2\pi
j}\cancel{e^{2\pi j }} d\nu = \sum_{n=-\infty}^{+\infty} c_n \int_0^1 e^{2\pi j
\nu} d\nu = \\
&= \begin{cases}
\text{se } n = 0 \rightarrow \int_0^1d\nu = 1 \\
\text{se } n\neq0 \rightarrow 0
\end{cases}
\end{aligned}
$$


Facciamo i calcoli $:)$
$$
\frac{1}{2\pi j} \oint_{C_1(0)} f(z) z^{m-1} dz = c_{-m}
$$
$$
\frac{1}{\cancel{2\pi j} } \sum_{n=-\infty}^{+\infty} c_n \int_0^1
e^{j2\pi\nu(n+m\cancel{-1})}\cancel{2\pi j}\cdot \cancel{e^{2\pi j \nu}}d\nu =
\sum_{n=-\infty}^{+\infty}c_n\delta(n+m)
$$
in questo caso si ottiene la Delta di ``kronekker'', si dice che ``campiona''
in $-m$.

Dunque i termini della trasformata, i valori della funzione nella trasformata
??? sono proprio i coefficienti della parte non olomorfa della funzione
$$
X(z) = \sum_{n=0}^{\infty} x(n)z^{-n} = \sum_{n=0}^{\infty} c_{-n}z^{-n}
$$
Se si svolge l'integrale di un circuito che circonda lo zero si ottengono
dunque proprio i coefficienti della sequenza, questo giustifica la biunivocità
della trasformazione e la formula per ottenere i coefficienti
$$
x(n) = \frac{1}{2\pi j} \oint_{c(0)} X(z)z^{n-1} dz
$$

Se si esegue la FT su una sequenza in funzione di $\nu$ per recuperare i
coefficienti $x(n)$, ossia ricostruire la sequenza a partire dalla sua
trasformata
$$
X(\nu) = \sum_{n=0}^{+\infty} x(n) e^{-j2\pi \nu n}
$$
allora
$$
x(n) = \int_0^1 X(\nu) e^{j2\pi \nu n} d\nu
$$

$$
X(\nu) = \left.X(z)\right|_{z=e^{j2\pi \nu}}
$$
è possibile sostituire questa posizione e ricavare la precedente.


\section{Costi del calcolo della DFT}
Si ricorda la definizione di DFT su N punti
$$
X(k) = \sum_{n=0}^{N-1}x(n)e^{-j\frac{2\pi}{N}kn},\ k = 0,\dots,N-1
$$
è un'operazione vettoriale, si moltiplica ogni termine $x(n)$ per un numero
complesso, la moltiplicazione si scompone in almeno quattro termini.
Si supponga un processore con una frequenza di clock di 1GHz, si supponga che
per realizzare una moltiplicazione tra addendi complessi si impieghi un certo
tempo ad esempio $10nS$.
Devono essere eseguite $n$ operazioni complesse per ogni valore di $k$, si
stanno trascurando i costi della somma e altri costi computazionali minori.
Si eseguiranno $N^2$ operazioni complesse, ad esempio con un brano di 100.000
punti, allora $N^2$ sarà pari a $10^10$ ossia 10 miliardi di operazioni
elementari, il tempo necessario sarà pari a
$$
10^10 \cdot 10nS = 10^11 nS = 10^11 \cdot 10^-9 = 100s
$$

L'algoritmo DFT può essere implementato in maniera efficiente mediante una sua
variante denominata FFT (FastFourierTransform) che permette di eseguire un
numero pari a $n\cdot\log_2(N)$, nell'esempio precedente con $10^6$ campioni si
avrà un tempo pari a
$$
10^6\log_2(10^6) = 20 * 10^6 * 10nS = 200 mS
$$
L'algoritmo di FFT produce lo stesso identico risultato della DFT ma
nell'esecuzione del calcolo si nota una ridondanza che viene eliminata dalla
FFT.

Si introduce la seguente definizione
$$
W_N = e^{j\frac{2\pi}{N}}; X(k) = \sum_{n=0}^{N-1} x(n) W_N^{-kn},\
k=0,\dots,N-1
$$
dunque
$$
X(k) = \sum_{n=0}^{\frac{N}{2}-1}x(2n) W_N^{-k2n} + \sum_{n=0}^{\frac{N}{2}-1}
x(2n+1)W_N^{-k(2n+1)},\ k=0,\dots,N-1
$$
Si è divisa la sommatoria in termini pari e dispari.

Con un cambio di notazione si pone il 2 al denominatore della $N$.

$$
x(k) = \sigma_{n=0}^{\frac{N}{2}-1} x_p(n) W_{\frac{N}{2}}^{-kn} +
W_N^{-K}\sum_{n=0}^{\frac{N}{2}-1}x_D(n) W_{\frac{N}{2}}^{-kn},\ k=0,\dots,N-1
$$
Dunque il primo termine è proprio una DFT su $N/2$ punti, allo stesso modo il
secondo termine con coefficienti dispari.

Se si divide la sequenza $x(n)$ nella sottosequenza pari e in quella dispari,
si può parallelizzare il processo, avendo il calcolo di due DFT su $N/2$ punti
in contemporanea.

Il coefficiente $W_N^{-k} = e^{-j\frac{2\pi}{N}k}$ è un fasore, per
$k\in[0,N/2]$ si sposta nel primo e secondo quadrante del piano complesso.
Viceversa $W_N^{-\left(\frac{N}{2}-1\right)} = -W_N^{-1}$.
Dunque per ottenere i primi $N/2-1$ valori della sequenza si moltiplica la DFT
dei termini pari per i termini dispari moltiplicati per $W_N^{-kn}$, per
calcolare invece i termini per $n\in[N/2, N-1]$ si moltiplicano per $-1$ i
termini dispari e sottratti a quelli pari.


%Qui manca qualcosa
Si può dunque lavorare su quattro sottosequenze con ovvia notazione $x_{pp},
x_{pp}, x_{dp}, x_{dd}$.
Si possono eseguire quattro DFT su blocchi con $N/4$ punti, con il grafo
identico al precedente.

Si supponga di avere $N=8$ dunque 8 linee, quindi 3 stadi di calcolo
$\log_2(8)=3$
Lo schema di calcolo detto a ``farfalla'' è sempre dello stesso tipo.

Il riordinamento dei campioni viene effettuato mediante la regola del
bit-reverse 100>001, 110>011 eccetera...

