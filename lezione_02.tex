
\section{Approfondimenti sul BUS IEEE 488 - GPIB}
Presenta 5 linee di GIM tra le quali
\begin{itemize}
 \item IFC
 \item REN
 \item ATN
 SRQ
 EOI
\end{itemize}
tre linee per la sincronizzazione che seguono il protocollo ``handshake'',
8 linnee per la trasmissione dati e 8 linee di riferimento (GND).

Un esempio di processo di misura da automatizzare è ancora il collaudo di un
doppio bipolo che realizza un filtro passa basso, il banco è costituito da tre
strumenti e un controller.
Il controller è un PC con interfaccia IEEE488, un counter a due canali, una
sorgente a singolo canale e un multimetro digitale a due canali.
Anche gli strumenti, come il PC sono dotati di interfaccia.

Il counter serve a misurare lo sfasamento tra i segnali isofrequenziali, il
multimetro misura il guadagno introdotto dal filtro, solitamente un guadagno
unitario nella banda passante che si riduce man mano entrando nella frequenza
di transizione, oltre la frequenza di taglio.

Il controller attesta la linea IFC e subito dopo da un secondo comando
universale unilinea REN in cui specifica alle unià periferiche che riceveranno
i comandi tramite interfaccia e non tramite pannello. Si attesta poi la linea
ATN e la stazione di misura viene posta in regime ``command mode''.

Il controller in questo funzionamento assegna dei ruoli agli strumenti mediante
un meccanismo di indirizzamento, ad ogni strumento sarà assegnato un indirizzo,
un intero che va da 1 a 30, per specificare quest'indirizzo si utilizzano 5 bit.

Con 5 bit si realizzerebbero $2^5$ combinazioni ma l'indirizzo 0 è riservato al
controller mentre l'indirizzo 31 è necessario all'abilitazione di un indirizzo
secondario, come ad esempio un modulo particolare di un certo strumento.

Sono comunque poche le applicazioni che necessitano del meccanismo di
indirizzamento secondario.

Si assegni alla sorgente l'indirizzo 1, al multimetro l'indirizzo 2 e al
counter l'indirizzo 8.
Per configurare inizialmente la sorgente va inviato un comando in regime
``device mode'' comprensibile solo alla sorgente, dopo averle assegnato un
ruolo, si riserva il bus ad una comunicazione dedicata tra un talker e un
listener.

Per assegnare il ruolo di listener alla sorgente si utilizzano lel linee dati
usando i 5 bit meno significativi l'indirizzo dello strumento, assegnando un
ruolo mediante il sesto e il settimo bit, l'ottavo bit rispetta la regola della
parità.

Il controller assegna il ruolo di listener al dispositivo con indirizzo 1

\verb|00100001 - A - !|

Si configura il controller come listener
\verb|11000000 - \@ - MTA0|

Si rilascia infine la inea attention (ATN) passando al regime device mode, il
counter e il multimetro non hanno visto il loro indirizzo, restando in idle.

La comunicazione sul BUS prevede sempre un solo talker ma può avere più
listener a patto che comprendano lo stesso linguaggio.

Gli strumenti di misura hanno sempre in allegato dei manuali, o una
documentazione con una sezione riservata alla programmazione dello strumento.

Un comando che solitamente si usa per configurare la sorgente è il comando

\verb|APPLy:SINusoidal 10,1000,0|

Nella sintassi estesa si può aggiungere l'unità di misura dei valori.
La sintassi contratta è composta solo dalle parti in maiuscolo, la sintassi è
decisa dal costruttore.
Si potrebbe in teoria assegnare un comando per ciascun carattere, riducendo il
numero di caratteri da inviare.

La velocità di throughput può raggiungere velocità pari al megabyte/s.
Lo standard richiede di non utilizzare cavi di lunghezza maggiori.

Rendere il linguaggio dello strumento stringato si è riscontrato che i comandi
per attivare le funzioni sono più semplici da memorizzare.
La scelta dei costruttori è stata quella di configurare i comandi da renderli
facilmente memorizzabili.

Durante la configurazione, la sorgente inizia a a generare il segnale stabile,
rendendo l'inerfaccia indipenente.

All'invio di ogni byte in device mode, si attiva una procedura di
configurazione per lo scambio dei dati con conseguente sincronizzazione

Se lo strumento riceve un comando errato, va in errore.

Il talker configura le linee dati utilizzando la procedura di handshake per
sincronizzare la comunicazione, quando ha terminato il messaggio, invia l'EOI,
end of ``I''

Terminata la comunicazione dal sistema sorgente, a talker torna il possesse oel
bus.

In regime command mode il controller associa il ruolo di listener
all'indirizzo 2

\verb|00100010|
e a ancora il ruolo di listener a sé stesso, vanno ora inviati dei comandi di
\textit{deselezione} degli strumenti precedentemente attivati, ad esempio si
invia il seguente comando per deselezionare i listener precedentemente attivati

\verb|00111111 - ?|

oppure per deselezionare i talker con il comando ``\_''

\verb|01011111 - _|

Per ottenere il risultato dal multimetro dopo averlo impostato come listener si
invia un comando di QUERY, terminato da punto interrogativo (?)

\verb|MEAS:CH1:VAC?|
Lo strumento deve conservare il risultato perché dopo una QUERY allo strumento
verrà assegnato il ruolo di parlatore e invierà il risultato al listener, il
ruolo di parlatore al dispositivo 2 si invia con il carattere B, ancora una
volta il controllo si imposta come listener con il carattere ``barra
spaziatrice''

\verb|01000010 - B|

Quando il multimetro termina la comunicazione si ritorna in regime di command
mode, si ripete la procedura per rieseguire la misura sul canale 2 (\verb|CH2|).

Lo stesso processo si ripete per il counter per eseguire la misura di
sfasamento, deseleziona precedenti listener e talker, invia una QUERY al
counter dopo averlo impostato come listener, dopo si imposta come talker per
permettergli l'invio del risultato.



Quelli presentati sono comandi multi linea universali, perché utilizzano più
linee per comunicare con i dispositivi, possono essere specifici se hanno un
indirizzo specifico e si riferiscono ad un singolo dispositivo oppure possono
essere aspecifici se inviano un comando a tutti i dispositivi come la
deselezione.


