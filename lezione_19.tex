
\section{Rumore di quantizzazione}
Tale rumore è un processo aleatorio, dunque si assegna solitamente a tale
funzione una variabile aleatoria, parte di un processo limitato e a media nulla?

$$
e = x - Q(x)
$$
Si asusme solitamente il modello di variabile aleatoria uniforme (pdf), data la
definizione di densità l'integrale della pdf deve essere unitario, allora la
sua ampiezza sarà pari a $1/\Delta$ con $\Delta$ l'ampiezza dell'intervallo.

La varianza è csì definita:
$$
\sigma^2_e = \infint (e-\mu)^2 pdf(e) de = \int_{-\Delta/2}^{\Delta/2} e^2 \frac{1}{\Delta} de =\left. \frac{1}{\Delta} \frac{e^3}{3} \right|_{-\Delta/2}^{\Delta/2} = 
\frac{1}{3\Delta} \left(\frac{\Delta^3}{8}+\frac{\Delta^3}{8}\right) = 
\frac{\Delta^2}{12}
$$

Si è assunto implicitamente che il rumore di quantizzazione sia stazionario, ovvero che le sue caratteristiche non dipendono dal tempo, se non fosse stato tale si sarebbe dovuta considerare una "pdf" dell'errore e del tempo.

Si analizza il processo di trasformazione di un convertitore analogico-digitale 
(ADC), si ha un processo di campionamento e quantizzazione, il processo di quantizzazione è sede di un rumore additivo $e(t)$

Per capire l'impatto del rummore di quantizzazione sul segnale si definisce una
metrica chiamata SNR (\textit{Signal-to-Noise-Ratio}) espressa in deciBel
$$
SNR \stackrel{\Delta}{=} 10\log_{10}\frac{\sigma_x^2}{\sigma_e^2} = 
20\log_{10}\frac{\sigma_x}{\sigma_e} = 10\log_{10}\frac{P_x}{P_e} = 
20\log_{10}\frac{V_{x,rms}}{V_{e,rms}}
$$

La potenza del rumore di quantizzazione è dunque
$$
\sigma_e^2 = \frac{\Delta^2}{12}
$$
con $\Delta = \frac{R}{2^b}$, $R$ è il range del convertitore e $b$ il numero 
di bit disponibili al convertitore per codificare il quanto (risoluzione).
Si sostituisce il passo $\Delta$ nella definizione di $\sigma_e$
$$
\sigma_e^2 = \frac{\Delta^2}{12} = \frac{1}{12}\left(\frac{R}{2^b}\right)^2 = \frac{R^2}{12\cdot 2^{2b}}
$$
Di conseguenza l'SNR definito precedentemente risulta pari a
$$
SNR = 10\log_{10}\frac{\sigma_x^2}{\sigma_e^2} = 10\log_{10}\left(\frac{\sigma_x^2}{R^2}12(2^2)^b\right) = 10\log_{10} 4^b + 10\log_{10}(12) +
10\log_{10}\left(\frac{\sigma_x^2}{R}\right)^2
$$
Approssimando i logaritmi si ottiene
$$
SNR = b\cdot 6.02 + 10.8 + 20\log_{10}\left(\frac{\sigma_x}{R}\right)
$$

Si considera ad esempio un segnale siusoidale di ampiezza $R/2$, il valore \textit{RMS} sarà $\frac{R}{2\sqrt{2}}$, il cui quadrato è $\frac{R^2}{8}$, si ripete la stima del SNR, ottenendo:
$$
SNR = 10\log_{10} \frac{x_{rms}^2}{e^2_{rms}} = 10\log_{10}\frac{R^2}{8R^2}12\cdot 4^b = 10\log_{10}4^b + 10\log_{10}(\frac{12}{8}) =
 b\cdot 6.02 + 1.73\ dB
$$
per un convertitore ad 8-bit il valore di SNR sarà $SNR \leq 50\ dB$, aggiungendo ad esempio un bit si incrementa il rapporto di circa $6\ dB$.

Se il convertitore è reale, si definisce la \textit{SINAD} ovvero il "signal to noise and distortion", si aggiunge un contributo di distorsione dovuto alla non uniformità della quantizzazione.

Se si fornisce ad un convertitore un segnale sinusoidale, accertandosi di evitare
il sovraccarico, tutti i valori ottenuti possono essere elaborati effettuando ad esempio un sinnusoidal fitting, si può valutare dunque il SINAD, sui dati 
quantizzati si può sottrarre il segnale descritto dal modello, in questo modo è possibile ottenere una misura del segnale rappresentativo del rumore e della distorsione $\varepsilon_n$, la potenza del rumore sarà
$P_e = \frac{1}{M}\sum_{n=1}^{M} \varepsilon^2_n$.
Il SINAD del convertitore si assume prossimo all'SNR entro la banda del convertitore per poi decrescere rapidamente superata la banda.

Vanno confrontati i grafici dei SINAD forniti dal produttore per valutare
la bontà di un convertitore, non conta solo il numero di bit e non è detto che un convertitore, ad esempioa 12 bit sia migliore di uno ad 8.

Le prestazioni possono essere espresse in modo equivalente con un ulteriore parametro chiamato ENOB (\textit{Effective number of bit}), numero effettivo di bit è un numero equivalente che permette di ricavare una relazione tra il SINAB e l'SNR, ovvero:
$$
SNR = 6.02b + 1.73 \rightarrow SINAD = 6.02\cdot ENOB + 1.73
$$
viene stimato sperimentalmente, non è pari fisicamente ad un numero di bit dato che può non essere intero, l'ENOB è sempre minore di $b$.

La potenza di un convertitore ideale:
$$
e^2_{rms} = \frac{\Delta^2}{12} = \frac{\left(\frac{R}{2^b}\right)^2}{12}
$$
aggiungendo la distorsione si ricava un $\Delta'>\Delta$ equivalente
$$
e^2_{rms+dist} = \frac{(\Delta')^2}{12} = \frac{\left(\frac{R}{2^{ENOB}}\right)^2}{12}
$$
$$
e^2_{tot,rms} = \frac{R}{2^{ENOB}\sqrt{12}} \frac{\Delta}{\Delta} 
$$
ma $\frac{R}{\Delta} = 2^b$ dunque sostituendo
$$
e^2_{tot,rms} = \frac{\Delta}{2^{ENOB}\sqrt{12}}2^b \Rightarrow 2^{ENOB} = 
2^b\frac{\frac{\Delta}{\sqrt{12}}}{e_{tot,rms}} \Rightarrow ENOB = \log_2 2^b + 
\log_2 \frac{\frac{\Delta}{\sqrt{12}}}{e_{tot,rms}} 
$$
in conclusione
$$
ENOB = b - \log_2 \frac{e^{reale}_{tot,rms}}{\frac{\Delta}{\sqrt{12}}}
$$
l'argomento del logaritmo è sempre maggiore di uno, dato dal rapporto del rumore
di un convertitore reale rispetto ad un convertitore ideale.

Anche l'ENOB come il SINAD è una metrica che decade all'aumentare della frequenza.

