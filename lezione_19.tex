
\section{Rumore di quantizzazione}
Tale rumore è un processo aleatorio, dunque si assegna solitamente a tale
funzione una variabile aleatoria, parte di un processo limitato e a media nulla?

$$
e = x - Q(x)
$$
Si asusme solitamente il modello di variabile aleatoria uniforme (pdf), data la
definizione di densità l'integrale della pdf deve essere unitario, allora la
sua ampiezza sarà pari a $1/\Delta$ con $\Delta$ l'ampiezza dell'intervallo.

La varianza è csì definita:
$$
\sigma^2_e = \infint (e-\mu)^2 pdf(e) de = \int_{-\Delta/2}^{\Delta/2} e^2 \frac{1}{\Delta} de =\left. \frac{1}{\Delta} \frac{e^3}{3} \right|_{-\Delta/2}^{\Delta/2} = 
\frac{1}{3\Delta} \left(\frac{\Delta^3}{8}+\frac{\Delta^3}{8}\right) = 
\frac{\Delta^2}{12}
$$

Si è assunto implicitamente che il rumore di quantizzazione sia stazionario, 
ovvero che le sue caratteristiche non dipendono dal tempo, se non fosse stato 
tale si sarebbe dovuta considerare una "pdf" dell'errore e del tempo.

Si analizza il processo di trasformazione di un convertitore analogico-digitale 
(ADC), si ha un processo di campionamento e quantizzazione, il processo di 
quantizzazione è sede di un rumore additivo $e(t)$

Per capire l'impatto del rummore di quantizzazione sul segnale si definisce una
metrica chiamata SNR (\textit{Signal-to-Noise-Ratio}) espressa in deciBel
$$
SNR \stackrel{\Delta}{=} 10\log_{10}\frac{\sigma_x^2}{\sigma_e^2} = 
20\log_{10}\frac{\sigma_x}{\sigma_e} = 10\log_{10}\frac{P_x}{P_e} = 
20\log_{10}\frac{V_{x,rms}}{V_{e,rms}}
$$

La potenza del rumore di quantizzazione è dunque
$$
\sigma_e^2 = \frac{\Delta^2}{12}
$$
con $\Delta = \frac{R}{2^b}$, $R$ è il range del convertitore e $b$ il numero 
di bit disponibili al convertitore per codificare il quanto (risoluzione).
Si sostituisce il passo $\Delta$ nella definizione di $\sigma_e$
$$
\sigma_e^2 = \frac{\Delta^2}{12} = \frac{1}{12}\left(\frac{R}{2^b}\right)^2 = \frac{R^2}{12\cdot 2^{2b}}
$$
Di conseguenza l'SNR definito precedentemente risulta pari a
$$
SNR = 10\log_{10}\frac{\sigma_x^2}{\sigma_e^2} = 10\log_{10}\left(\frac{\sigma_x^2}{R^2}12(2^2)^b\right) = 10\log_{10} 4^b + 10\log_{10}(12) +
10\log_{10}\left(\frac{\sigma_x^2}{R}\right)^2
$$
Approssimando i logaritmi si ottiene
$$
SNR = b\cdot 6.02 + 10.8 + 20\log_{10}\left(\frac{\sigma_x}{R}\right)
$$

Si considera ad esempio un segnale siusoidale di ampiezza $R/2$, il valore 
\textit{RMS} sarà $\frac{R}{2\sqrt{2}}$, il cui quadrato è $\frac{R^2}{8}$, si 
ripete la stima del SNR, ottenendo:
$$
SNR = 10\log_{10} \frac{x_{rms}^2}{e^2_{rms}} = 10\log_{10}\frac{R^2}{8R^2}
12\cdot 4^b = 10\log_{10}4^b + 10\log_{10}(\frac{12}{8}) =
 b\cdot 6.02 + 1.73\ dB
$$
per un convertitore ad 8-bit il valore di SNR sarà $SNR \leq 50\ dB$, 
aggiungendo ad esempio un bit si incrementa il rapporto di circa $6\ dB$.

Se il convertitore è reale, si definisce la \textit{SINAD} ovvero il "signal to 
noise and distortion", si aggiunge un contributo di distorsione dovuto alla non 
uniformità della quantizzazione.

Se si fornisce ad un convertitore un segnale sinusoidale, accertandosi di 
evitare
il sovraccarico, tutti i valori ottenuti possono essere elaborati effettuando 
ad esempio un sinnusoidal fitting, si può valutare dunque il SINAD, sui dati 
quantizzati si può sottrarre il segnale descritto dal modello, in questo modo è 
possibile ottenere una misura del segnale rappresentativo del rumore e della 
distorsione $\varepsilon_n$, la potenza del rumore sarà
$P_e = \frac{1}{M}\sum_{n=1}^{M} \varepsilon^2_n$.
Il SINAD del convertitore si assume prossimo all'SNR entro la banda del 
convertitore per poi decrescere rapidamente superata la banda.

Vanno confrontati i grafici dei SINAD forniti dal produttore per valutare
la bontà di un convertitore, non conta solo il numero di bit e non è detto che 
un convertitore, ad esempioa 12 bit sia migliore di uno ad 8.

Le prestazioni possono essere espresse in modo equivalente con un ulteriore 
parametro chiamato ENOB (\textit{Effective number of bit}), numero effettivo di 
bit è un numero equivalente che permette di ricavare una relazione tra il SINAB 
e l'SNR, ovvero:
$$
SNR = 6.02b + 1.73 \rightarrow SINAD = 6.02\cdot ENOB + 1.73
$$
viene stimato sperimentalmente, non è pari fisicamente ad un numero di bit dato che può non essere intero, l'ENOB è sempre minore di $b$.

La potenza di un convertitore ideale:
$$
e^2_{rms} = \frac{\Delta^2}{12} = \frac{\left(\frac{R}{2^b}\right)^2}{12}
$$
aggiungendo la distorsione si ricava un $\Delta'>\Delta$ equivalente
$$
e^2_{rms+dist} = \frac{(\Delta')^2}{12} = \frac{\left(\frac{R}{2^{ENOB}}\right)^2}{12}
$$
$$
e^2_{tot,rms} = \frac{R}{2^{ENOB}\sqrt{12}} \frac{\Delta}{\Delta} 
$$
ma $\frac{R}{\Delta} = 2^b$ dunque sostituendo
$$
e^2_{tot,rms} = \frac{\Delta}{2^{ENOB}\sqrt{12}}2^b \Rightarrow 2^{ENOB} = 
2^b\frac{\frac{\Delta}{\sqrt{12}}}{e_{tot,rms}} \Rightarrow ENOB = \log_2 2^b + 
\log_2 \frac{\frac{\Delta}{\sqrt{12}}}{e_{tot,rms}} 
$$
in conclusione
$$
ENOB = b - \log_2 \frac{e^{reale}_{tot,rms}}{\frac{\Delta}{\sqrt{12}}}
$$
l'argomento del logaritmo è sempre maggiore di uno, dato dal rapporto del rumore
di un convertitore reale rispetto ad un convertitore ideale.

Anche l'ENOB come il SINAD è una metrica che decade all'aumentare della frequenza.

È possibile utilizzare degli approcci di elaborazione numerica che consentono 
di incrementare la risoluzione del convertitore, barattando frequenza e 
risoluzione.
Si consideri collegata in serie una macchina che elabori i segnali acquisiti 
dal convertitore, in streaming, è possibile implementare un paradigma di 
elaborazione
che prevede un filtro passa-basso (FIR) e uno stadio di \textit{decimazione}. 
Il filtro FIR può essere un filtro a media mobile, aritmetica o pesata ecc... È 
possibile barattare l'oversampling per la risoluzione, il contenuto spettrale 
di un segnale di rumore di quantizzazione introdotto dall'ADC è tipicamente 
uniformemente distribuito su tutto lo spettro fino ad $M\cdot f_s/2$. Ciò 
significa che se si sceglie una certa frequenza di campionamento $f_s/2$, allora
la potenza del rumore di quantizzazione
$$
e^2_{rms} = \frac{\Delta^2}{12} 
$$
questa densità ha una diversa ampiezza se si lavora con una frequenza $f_s$ più 
alta o 
più bassa, si abbatte il rumore se si lavora con una frequenza di campionamento 
più elevata.

Sia dato un certo segnale sinusoidale da esaminare a frequenza nota, si ha una 
dinamica molto più elevata usando una frequenza di campionamento maggiore, per 
quanto riguarda il rapporto SNR in realtà non c'è alcuna variazione (stessa 
cosa per l'ENOB)

Analizzando un segnale con una frequenza di campionamento molto più elevata del 
segnale stesso, si ottengono numerosi campioni "superflui", il numero di 
campioni memorizzabili è finito. Accoppiando l'uso di un filtro passa-basso ed 
una elevata frequenza di campionamento è possibile attenuare efficacemente 
l'errore.
$$
SNR = 10\log_{10} \frac{P_x}{\frac{\sigma_e^2}{M}} = 10\log_{10} \frac{P_x}{\sigma_e^2} + 10\log_{10}M
$$
$M$ è solitamente un multiplo di 2, raddoppiando ad esempio la $f_s$ si 
raddoppia il rapporto SNR o equivalentemente si incrementa la risoluzione di 
mezzo bit, quadruplicando la $f_s$ si ottiene $p = 2$
$$
10\log_{10}2^p = p\cdot \log_{10}2 = p\cdot 3.01
$$
Il limite superiore alla frequenza è la saturazione del convertitore.

In molti oscilloscopi la modalità di sovracampionamento e decimazione è
offerta automaticamente dalla macchina ed eseguita in streaming, non è 
necessario dunque riversare in memoria tutti i campioni acquisiti ad una $f_s$ 
più elevata ma solo i campioni utili, che non si ripetono, vengono memorizzati, 
dunque la memorizzazione avviene ad una frequenza più bassa rispetto alla 
frequenza di campionamento, ciò permette di ottenere un risparmio in memoria.
Alcuni nomi commerciali per questa modalità sono: \textit{HiRes}, 
\textit{EnhRes}... l'algoritmo della \textit{Tektronix} utilizza la media 
mobile.
La rappresentazione dei valori è \textit{estesa} ovvero con 8 bit si possono 
avere rappresentazioni a 16 bit per rappresentare i numeri decimali, o comunque 
i valori intermedi tra i codici. 
I valori non saranno più [-127,127] ma accodando un byte nella rappresentazione, che equivale a moltiplicare per 128, i valori diventeranno
65536 ovvero [-32767,32767], in realtà lo step tra un valore e l'altro non sarà 
più di 1 ma di 256, non diventeranno mai significativi a meno che non si 
abilita una modalità di acquisizione \textit{HiRes} allora effettuando la media 
aritmetica tra due campioni nella codifica ad 8 bit si trova un valore 
intermedio rappresentato nella codifica a 2 byte (16bit).

Un'altra modalità di acquisizione che sfrutta la codifica a 2 byte è la 
modalità di acquisizione \textit{Average}, non si sovrascrivono i campioni ma 
vengono mediati tra schermate successive in memoria, dunque diventa 
significativo il byte aggiunto nella rappresentazione a 2 byte.
